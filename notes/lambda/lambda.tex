\documentclass[a4paper,10pt]{article}

\usepackage{listings}
\usepackage[utf8]{inputenc}
\usepackage{stmaryrd}
\usepackage{yfonts}
\usepackage{bm}

\usepackage[margin=1.00in]{geometry}
\usepackage{cite}
\usepackage{titling}
\usepackage{amsmath}
\usepackage{indentfirst}
\usepackage[textwidth=3.7cm]{todonotes}
\setlength{\marginparwidth}{3.7cm}
\usepackage{etoolbox}
\lstset{breaklines}

\usepackage[compact]{titlesec}         % you need this package
\titlespacing{\section}{0pt}{0pt}{0pt} % this reduces space between (sub)sections to 0pt, for example
\titlespacing{\subsection}{0pt}{0pt}{0pt} % this reduces space between (sub)sections to 0pt, for example
\titlespacing{\subsubsection}{0pt}{0pt}{0pt} % this reduces space between (sub)sections to 0pt, for example
\AtBeginDocument{%                     % this will reduce spaces between parts (above and below) of texts within a (sub)section to 0pt, for example - like between an 'eqnarray' and text
    \setlength\abovedisplayskip{0pt}
    \setlength\belowdisplayskip{0pt}
}

\setlength{\droptitle}{-5em} % This is your set scream
\setlength{\parindent}{1em}
\setlength{\parskip}{0.5em}
\setlength{\parindent}{0in}


\title{$\lambda$-Calculus and Lawvere's Fixed Point Theorem}
\date{}

\begin{document}
\maketitle
\vspace{-20mm}
Lawvere's Fixed Point theorem generalises many important theorems across
mathematical logic and the theory of computation. This note outlines a
relationship between Lawvere's fixed point theorem and models of the untyped
lambda calculus. All new theorems in this paper have been proven in the theorem
prover \verb|Agda|. All subsidiary information in this note can be found in
Barendregt's  "The Lambda Calculus - Its Syntax and Semantics" with a focus on
section 5 on Models.

As a reminder, Lawvere's fixed point theorem states that, in some CCC, $C$, if
there exists a point-surjective function $\phi \, : \, A \rightarrow B^A$ for
some $A$ and $B$, all endomorphisms on $B$ have a fixed point.

A corollary of this is that, if there exists a reflexive object with a retract,
an object, $U$,  with a pair of arrows $u \, : \, U \rightarrow U^U$, $u^{-1} \,
: \, U^U \rightarrow U$ with $u \circ u^{-1} = id$, all endomorphisms on $U$
have a fixed point.

This has a close and distinct relation to models of the lambda calculus. From
Barendregt "\ldots for the construction of a $\lambda$-calculus model it
is sufficient to have an object $D$ in a CCC such that $D^D$ is a retract of
$D$."

In light of the simple corollary to Lawvere's fixed point theorem it can be seen
that every $\lambda$-calculus model has fixed point for every endomorphism for
the relevant interpretation of its reflexive object. Papers that explore the
consequences of lawvere's fixed point theorem sometimes claim that Lawvere's
fixed point theorem implies the existence of a fixed point combinator in the
untyped $\lambda$-calculus. One might imagine that this is the case given the
corollary outlined above and Barendregt's statements however the story is
slightly more complicated. Lawvere's fixed point theorem can be used not to
derive the existence of a fixed point combinator in the untyped
$\lambda$-calculus but instead the first fixed point theorem, a simple
consequence of the existence of fixed point combinators.

The thesis that all models of the $\lambda$-calculus arise from CCCs with a
reflexive object arises from the relationship between models of the
$\lambda$-calculus, named $\lambda$-algebras, and CCCs with a reflexive object.
$\lambda$-algebras are models that satisfy  all provable equations of the
$\lambda$-calculus and are an entirely equational theory. Via a transformation
called the Karoubi Envelope, every $\lambda$-algebra can be turned into a CCC
with a reflexive object. A transformation exists in the reverse direction that
takes every CCC with a reflexive object to a $\lambda$-algebra. Going from
algebra to CCC to algebra again produces an isomorphic algebra to the original.

The definition of $\lambda$-algebras will briefly be summarised below

$M = (X \, , \: \cdot)$ is an applicative structure if $\cdot$ is a binary operation on
$X$.


A combinatory algebra is an applicative structure with distinguished elements
$k$ and $s$ which satisfy.

\begin{align*}
kxy &= x \\
sxyz &= xz (yz)
\end{align*}

With this the terms and pure equational theory of combinatory logic is lifted to
an arbitrary underlying carrier set. From the study of combinatory logic, it is
known that further axioms are required for the equational theory of combinatory
logic to create an equivalence between this and the equational theory $\lambda$.
These axioms, as discovered by Curry, are known as $A_{\beta}$ and can be found
in the Barendregt. These axioms are precisely the axioms added to extend
combinatory algebras to $\lambda$-algebras. Given this terms and proofs in
lambda algebras can be given using lambda terms which are then converted to
terms in combinatory logic and then lifted to lambda algebras.


An alternate formulation of $\lambda$-algebras are the so-called environment or
syntactic models. Syntactic applicative structures extend original ones with a
set of semantic brackets $\llbracket  \, \, \rrbracket$ that allow for an
inteprataton of the $\lambda$-calculus in the underlying set of the applicative
structure. Constraints on the semantic brackets are outlined further in
Barendregt. Barendregt further outlines a notion of satisfaction for syntactic
applicative structures.

Syntactic $\lambda$-algebras are the syntactic applicative structures,
\textfrak{M} such that

\begin{align*}
    \bm{\lambda} \vdash M = N \Rightarrow \textfrak{M} \models \llbracket M
    \rrbracket = \llbracket N \rrbracket
\end{align*}
\\

where $\bm{\lambda}$ is the equational theory of the $\lambda$-calculus.
Syntactic $\lambda$-algebras are equivalent to the earlier formulation of
$\lambda$-algebras.

Every (locally small) CCC, $C$, with a reflexive object, $D$, can be turned into
a syntactic $\lambda$-algebra as follows: Let the underlying set be the points
to $D$, written $|D|$, let the operation $\star$ on points $a$ and $b$ where $F
: D \rightarrow D^D$ is the morphism guaranteed by the reflexive $D$

\begin{align*}
    a \star b = eval \circ \langle \, F \times id \, \rangle \circ \langle \, a \,
    , \, b \, \rangle
\end{align*}
\\
The semantic interpretation function is outlined in the Barendregt and the proof
of satisfaction is provided and not relevant here. Barendregt provides a proof
originally by Koymans that applying the karoubi envelope to a $\lambda$-algebra
and performing the above transformation back produces an isomorphic
$\lambda$-algebra thus establishing that every $\lambda$-algebra can be obtained
from a CCC with a reflexive object.

The earlier established corollary that all categorical models of the
$\lambda$-calculus have endomorphsisms on their reflexive object will now be
used to derive the first fixed point theorem in the $\lambda$-calculus in all
models of the $\lambda$-calculus.

The first fixed point theorem in the $\lambda$-calculus states that all lambda
terms have a fixed point under application. The binary operation of applicative
structures is meant to represent application and thus the required statement is
that, for all $\lambda$-models $(A \, , \, \bullet \, , \, \llbracket \,
\rrbracket \, )$,
$\forall \, f \in A \, , \, \exists p \in A \, s.t. \, f \, \bullet \, p = p$

Given that every $\lambda$-algebra can be obtained from a CCC with a reflexive
object we can consider the proof for the transformation from these to a
$\lambda$-algebra i.e. that for all CCCs with a reflexive object $D$, that
for all points, $a$ to $D$ that there exists a $b$ such that $a \star b = b$.
The proof of this is very similar to the proof of lawvere's fixed point theorem.

Let $t = F \circ a $ be a morphism from $1 \rightarrow D^D$. This can be turned
into a morphism from $D \rightarrow D$ by pushing it first through the
adjunction for products and exponentials $\textrm{hom}(A \times B \, , \, C)
\cong \textrm{hom}(A \, , \, C^B)$ in the leftward direction which we will call
$uncurry$ and then through the isomorphism between $1 \times A \cong A$ which
will be denoted using a bar above the expression i.e. $u = \overline{uncurry\, (
F \circ a)}$ by the corollary to Lawvere's fixed point theorem in $CCCs$ with a
reflexive object $u$ has a fixed point which we call $p$. Simple algebra can now
be used to see that p is the fixed point of $a$ under $\star$

\begin{align*}
    a \star p &= eval \circ \langle F \times id \rangle \circ \langle a , p
    \rangle \\
    &= eval \circ \langle F \circ a \times id \rangle \circ \langle id , p
    \rangle \\
    &= uncurry \, (\, F \circ a ) \circ \langle id , p \rangle \\
    &= \overline{uncurry \, (F \circ a)} \circ p \\
    &= p
\end{align*}

Thus Lawvere's theorem can be seen to derive a significant result in the
$\lambda$-calculus. Lawvere's theorem may give some more insight into the theory
of the $\lambda$-calculus. CCCs with a reflexive object, $D$ are related to the
$\lambda$-algebras and further, CCCs that have an object $D$ with $D \cong D^D$
correspond to the so called $\lambda$-models which are all models of the
$\lambda$-calculus that satisfy the $\eta$-rule ($\lambda x .  M \, x = \lambda
x . N \, x  \Rightarrow M = N$). An (to my knowledge) outstanding question is
what the categorical model of combinatory algebras are.  Whilst we do not answer
this question we provide some motivating evidence that the necessary property
may be point-surjectivity or something similar. The underlying theory of
combinatory algebras is combinatory logic, a simple calculus of combinators that
are a model of computation. The minimal basis of combinators are $K$ and $S$
with the laws outlined earlier however many other standard combinators exist
(see the end of "To Mock a Mockingbird" by Raymond Smullyan). Three "standard"
combinators can be created in a CCC with a point-surjective $\varphi : D
\rightarrow D^D$. These combinators utilise the fact that $a \star b =
\overline{uncurry \, (\varphi \circ a)} \circ b$. If we are able to pick $a$
then we are able to recover any endomorphism on $D$. More precisely, for any $f:
D \rightarrow D$, this can be turned into a point to $D^D$ by pushing through
the other way in the $1 \times A \cong A$ isomorphism and exploiting the other
direction of the adjunction which we will call curry i.e. $f' = curry \, (
\,\underline{f} \,) : 1 \rightarrow D^D$. The point-surjectivity of $\varphi$
can now be used to find the equivalent $u$ such that $\varphi \circ u = f'$.
Considering $u \star b$ for any $b$ we get

\begin{align*}
    u \star b &= \overline{uncurry \, (\varphi \circ u \, )} \circ b \\
    &= \overline{uncurry \, ( \, f' \, ) } \circ b \\
    &= \overline{uncurry \, ( \, curry \, ( \, \underline{f} \, ) \, )} \circ b \\
    &= f \circ b
\end{align*}


A fairly easy combinator to construct is the identity combinator $\textbf{I} \, x =
x$ by taking $f$ in the above construction to be $id$ and calculating the
equivalent $u$.

Another useful combinator is the mockingbird or self-application operator,
$\textbf{M} x = x x$ by taking $f$ to be the following morphism of type $D
\rightarrow D$, $\textbf{M'} = eval \circ \langle \varphi \times id \rangle \circ
\delta$ and finding the equivalent $u$ and setting it to be \textbf{M}.

The final combinator we give a derivation for was found whilst attempting to
recover the \textbf{K} combinator and requires slightly more machinery.

Let $x = id : D \rightarrow D$, $y = curry \, ( \, \underline{x} \, ) : 1
\rightarrow D^D$ pick $z : 1 \rightarrow D$ s.t. $\varphi \circ z = y$ from the
point-surjectivity of $\varphi$ let $q = z \, \circ \, !_{D} : D \rightarrow D$
where $!_{D}$ is the terminal arrow from $D$. Taking $f$ as $q$ and deriving the
appropriate $u$ we get an interesting combinator. Calling the appropriate
$u$, \textbf{F}

\begin{align*}
    \textbf{F} \star a \star b &= (\textbf{F} \circ a) \star b \\
    &= eval \circ \langle \varphi  \times id \rangle \circ \langle q \circ a , b
    \rangle \\
    &= eval \circ \langle \varphi \times id \rangle \circ \langle z \, \circ \, !_{D}
    \, \circ \, a , \, b \rangle \\
    &= eval \circ \langle \varphi \circ z \times id \rangle \circ \langle !_{D} \,
    \circ \, a \, , \, b \rangle \\
    &= \overline{uncurry \, ( \varphi \circ z \, )} \circ b \\
    &= id \circ b \\
    &= b
\end{align*}
i.e. \textbf{F} selects the second of its two arguments. With some work perhaps
more combinators could be derived and, potentially, they will form a complete
basis for combinatory logic. Even if not , it is interesting to understand to
which computational world point-surjectivity corresponds.

\subsection{Karoubi Envelope Attempt}

Below is a failed attempt to see the consequences of lawvere's fixed pont
theorem after applying the Karoubi Envelope to the $\lambda$-algebra.

As stated earlier every $\lambda$-algebra can be turned into a ccc with a
reflexive object via the karoubi envelope which turns any additive category into
a pseudoabelian category. Let $C = (X, \cdot , k, s)$ be a $\lambda$-algebra the
karoubi envelope is as follows where $a \circ b = \lambda x . a (b x)$.

\begin{align*}
    &\textnormal{Objects}: \, \{ x \in X \, | \, x \circ x = x \} \\
    &\textnormal{Arrows}: \, \textnormal{Hom}(a, \, b) =  \{ f \in X \, | \, b
        \circ f
    \cdot a = f\} \\
    &\textnormal{Identity}: \, \textnormal{id}_{a} = a \\
    &\textnormal{Composition}: \, f \circ g
\end{align*}

This comes from considering $C$ as a monoid which is then categorified as a
category with a single object.

Proofs of the validity of all constructions can be found in (Koymans 1982). To
show cccsness.

\begin{align*}
    &\textnormal{Terminal}: \, t = \lambda x y . y \\
\end{align*}
\ldots

The reflexive object is \textbf{I} as $\textbf{I}^\textbf{I} = \textbf{1}$ where
$\textbf{I} = \lambda x.x$ and $\textbf{1} = \lambda x \, y . x \, y$ where
\textbf{1} is itself the arrow both ways between \textbf{1} and \textbf{I} and
$\textbf{1} \circ \textbf{1} = \textnormal{id}_{\textbf{1}}$.

Via the earlier corollary, every endomorphism on \textbf{I} has a fixed point.
Endomorphisms on \textbf{I} are $\lambda$-terms, $f$, such that $\textbf{I}
\circ f \circ \textbf{I} = f$. Expanding this out
\begin{align*}
    \textbf{I} \circ f \circ \textbf{I} &= \lambda x . \, \textbf{I} \, (f \, (\textbf{I}
    \, x)) \\
    &= \lambda x . \, f \, x
\end{align*}

For this to encompass all (any?) $\lambda$-terms our equational theory must
include the $\eta$-rule. Given that this is a construction is for
$\lambda$-algebras which need not have $\eta$ this rains on the hope that
Lawvere's fixed point theorem will correspond to the first recursion theorem for
the untyped $\lambda$-calculus.

A point to \textbf{I}, $p$, in the karoubi'd category corresponds to $\lambda$-terms such
that $\textbf{I} \circ p \circ t = p$

\begin{align*}
    \textbf{I} \circ p \circ t &= p \circ t \\
    &= \lambda x . p (t x) \\
    &= \lambda x . p (\lambda y . y) \\
    &= \lambda x . p \textbf{I} \\
    &= \textbf{K} (p \textbf{I})
\end{align*}

i.e some constant function.

With this in mind, Lawvere's fixed point theorem ends up representing a
reasonably strange theorem in the untyped $\lambda$-calculus i.e. for every
$\lambda$-term, $f$, that satisfies extensionality there exists a constant
$\lambda$-term, $u$, such that $\lambda x . f (u x) = u$. $u =
\textbf{K}(p\textbf{I})$ for some $p$.

\begin{align*}
    \lambda x . f (u x ) &= \lambda x . f(( \textbf{K} (p \textbf{I})) x) \\
    &= \lambda x . f(p \textbf{I}) \\
    &= \textbf{K} (f (p \textbf{I})) \\
    &= \textbf{K} (p \textbf{I})
\end{align*}



\end{document}


