\documentclass[a4paper,10pt]{article}

\usepackage{listings}
\usepackage[utf8]{inputenc}

\usepackage[margin=1.00in]{geometry}
\usepackage{cite}
\usepackage{titling}
\usepackage{amsmath}
\usepackage{indentfirst}
\usepackage[textwidth=3.7cm]{todonotes}
\setlength{\marginparwidth}{3.7cm}
\usepackage{etoolbox}
\lstset{breaklines}

\usepackage[compact]{titlesec}         % you need this package
\titlespacing{\section}{0pt}{0pt}{0pt} % this reduces space between (sub)sections to 0pt, for example
\titlespacing{\subsection}{0pt}{0pt}{0pt} % this reduces space between (sub)sections to 0pt, for example
\titlespacing{\subsubsection}{0pt}{0pt}{0pt} % this reduces space between (sub)sections to 0pt, for example
\AtBeginDocument{%                     % this will reduce spaces between parts (above and below) of texts within a (sub)section to 0pt, for example - like between an 'eqnarray' and text
    \setlength\abovedisplayskip{0pt}
    \setlength\belowdisplayskip{0pt}
}

\setlength{\droptitle}{-5em} % This is your set scream
\setlength{\parindent}{1em}
\setlength{\parskip}{0.5em}
\setlength{\parindent}{0in}


\title{$\lambda$-Calculus and Lawvere's Fixed Point Theorem}
\date{}

\begin{document}
\maketitle
\vspace{-20mm}
\section{Models}
The structure of Lawvere's theorem suggests an interplay with the Lambda
Calculus, not simply due to the relation between the lambda calculus,
propositional logic and computability.

As a reminder, Lawvere's fixed point theorem states that, in some CCC, $C$, if
there exists a point-surjective function $\phi \, : \, A \rightarrow B^A$ for
some $A$ and $B$, all endomorphisms on $B$ have a fixed point.

A corollary of this is that, if there exists a reflexive object with a retract,
an object, $U$,  with a pair of arrows $u \, : \, U \rightarrow U^U$, $u^{-1} \,
: \, U^U \rightarrow U$ with $u \circ u^{-1} = id$, all endomorphisms on $U$
have a fixed point.

This has a close and distinct relation to models of the lambda calculus. From
Barendregt (1984) "\ldots for the construction of a $\lambda$-calculus model it
is sufficient to have an object $D$ in a CCC such that $D^D$ is a retract of
$D$."

In light of the simple corollary to Lawvere's fixed point theorem it can be seen
that every $\lambda$-calculus model has fixed point for every endomorphism for
the relevant intepretation of its reflexive object. One might imagine that this
may in some way correspond to the Y-combinator for the Lambda Calculus however
under closer examination this yields some not so obvious results. Examining the
implications of this observation requires an examinations of models of the
$\lambda$-calculus.

The thesis that all models of the $\lambda$-calculus arise from CCCs with a
reflexive object arises from the isomorphism between the so-callled
$\lambda$-algebras and CCCs with a reflexive object. $\lambda$-algebras are
models that satisfy  all provable equations of the $\lambda$-calculus and are an
entirely equational theory. Via a process called the Karoubi Envelope, every
$\lambda$-algebra can be turned into a CCC with a reflexive object. The
implications of the corollary in this domain will be examined below.

The definition of $\lambda$-algebras will briefly be summarised below

\subsection{$\lambda$-Algebras}

$M = (X \, , \: \cdot)$ is an applicative structure if $\cdot$ is a binary operation on
$X$.


A combinatory algebra is an applicative structure with distinguished elements
$k$ and $s$ which satisfy.

\begin{align*}
kxy &= x \\
sxyz &= xz (yz)
\end{align*}

With this the terms and pure equational theory of combinatory logic is lifted to
an abitrary underlying carrier set. From the study of combinatory logic
(Barendregt 1984), it is known that further axioms are required for the
equational theory of combinatory logic to create an equivalence between this and
the equational theory $\lambda$. These axioms, as discovered by Curry, are known
as $A_{\beta}$ and can be found in any good textbook on combinatory logic. These
axioms are precisely the axioms added to extend combinatory algebras to
$\lambda$-algebras. Given this terms and proofs in lambda algebras can be given
using lambda terms which are then converted to terms in combinatory logic and
then lifted to lambda algebras.

As stated earlier every $\lambda$-algebra can be turned into a ccc with a
reflexive object via the karoubi envelope which turns any additive category into
a pseudoabelian category. Let $C = (X, \cdot , k, s)$ be a $\lambda$-algebra the
karoubi envelope is as follows where $a \circ b = \lambda x . a (b x)$.

\begin{align*}
    &\textnormal{Objects}: \, \{ x \in X \, | \, x \circ x = x \} \\
    &\textnormal{Arrows}: \, \textnormal{Hom}(a, \, b) =  \{ f \in X \, | \, b
        \circ f
    \cdot a = f\} \\
    &\textnormal{Identity}: \, \textnormal{id}_{a} = a \\
    &\textnormal{Composition}: \, f \circ g
\end{align*}

This comes from considering $C$ as a monoid which is then categorified as a
category with a single object.

Proofs of the validity of all constructions can be found in (Koymans 1982). To
show cccsness.

\begin{align*}
    &\textnormal{Terminal}: \, t = \lambda x y . y \\
\end{align*}
\ldots

The reflexive object is \textbf{I} as $\textbf{I}^\textbf{I} = \textbf{1}$ where
$\textbf{I} = \lambda x.x$ and $\textbf{1} = \lambda x \, y . x \, y$ where
\textbf{1} is itself the arrow both ways between \textbf{1} and \textbf{I} and
$\textbf{1} \circ \textbf{1} = \textnormal{id}_{\textbf{1}}$.

Via the earlier corollary, every endomorphism on \textbf{I} has a fixed point.
Endomorphisms on \textbf{I} are $\lambda$-terms, $f$, such that $\textbf{I}
\circ f \circ \textbf{I} = f$. Expanding this out
\begin{align*}
    \textbf{I} \circ f \circ \textbf{I} &= \lambda x . \, \textbf{I} \, (f \, (\textbf{I}
    \, x)) \\
    &= \lambda x . \, f \, x
\end{align*}

For this to encompass all (any?) $\lambda$-terms our equational theory must
include the $\eta$-rule. Given that this is a construction is for
$\lambda$-algebras which need not have $\eta$ this rains on the hope that
Lawvere's fixed point theorem will correspond to the first recursion theorem for
the untyped $\lambda$-calculus.

A point to \textbf{I}, $p$, in the karoubi'd category corresponds to $\lambda$-terms such
that $\textbf{I} \circ p \circ t = p$

\begin{align*}
    \textbf{I} \circ p \circ t &= p \circ t \\
    &= \lambda x . p (t x) \\
    &= \lambda x . p (\lambda y . y) \\
    &= \lambda x . p \textbf{I} \\
    &= \textbf{K} (p \textbf{I})
\end{align*}

i.e some constant function.

With this in mind, Lawvere's fixed point theorem ends up representing a
reasonably strange theorem in the untyped $\lambda$-calculus i.e. for every
$\lambda$-term, $f$, that satisfies extensionality there exists a constant
$\lambda$-term, $u$, such that $\lambda x . f (u x) = u$. $u =
\textbf{K}(p\textbf{I})$ for some $p$.

\begin{align*}
    \lambda x . f (u x ) &= \lambda x . f(( \textbf{K} (p \textbf{I})) x) \\
    &= \lambda x . f(p \textbf{I}) \\
    &= \textbf{K} (f (p \textbf{I})) \\
    &= \textbf{K} (p \textbf{I})
\end{align*}

The last line provides some hope for a partial recovery of the first recursion
theorem. Analysing what $p$ is, as guaranteed by LWFPT, may give sommay also be of use.e insight.


Analysing the other direction is a lot more fruitful

\section{Scott-Curry}
  - Diagonalization operator \\
  - Godel \\
  - Tarski \\
  - Rice \\
  - Computable functions \\

\end{document}


