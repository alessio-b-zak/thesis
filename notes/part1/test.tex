\documentclass[a4paper,12pt]{article}

\usepackage{listings}
\usepackage[utf8]{inputenc}

\usepackage[margin=1.00in]{geometry}
\usepackage{tikz-cd}
\usepackage{cite}
\usepackage{titling}
\usepackage{indentfirst}
\usepackage{amsmath}
\usepackage[textwidth=3.7cm]{todonotes}
\setlength{\marginparwidth}{3.7cm}
\usepackage{etoolbox}
\lstset{breaklines}

\usepackage[compact]{titlesec}         % you need this package
\titlespacing{\section}{0pt}{0pt}{0pt} % this reduces space between (sub)sections to 0pt, for example
\titlespacing{\subsection}{0pt}{0pt}{0pt} % this reduces space between (sub)sections to 0pt, for example
\titlespacing{\subsubsection}{0pt}{0pt}{0pt} % this reduces space between (sub)sections to 0pt, for example
\AtBeginDocument{%                     % this will reduce spaces between parts (above and below) of texts within a (sub)section to 0pt, for example - like between an 'eqnarray' and text
    \setlength\abovedisplayskip{-12pt}
    \setlength\belowdisplayskip{3pt}
}

\setlength{\droptitle}{-5em} % This is your set scream
\setlength{\parindent}{1em}
\setlength{\parskip}{0.8em}
\setlength{\parindent}{0in}
\date{}

\title{Temp\vspace{-3.5em}}

\begin{document}
\maketitle
A common notion in Category theory is that of a universal construction.
Universal constructions are common patterns that occur throughout mathematics
that aim to capture the essence of these patterns at the categorical level.
Before attempting to abstract these ideas it is useful to consider some
examples. Only the constructions most relevant to the heart of the thesis will
be discussed here.  After looking at the individual constructions we will look
at three ways of abstracting them and look at the links between them.

\section{Products}
Products exemplify a common construction in categories of combining the
structure of two objects (in some canonical way) within the category to produce
an object of the same category. In more concrete terms the product of two
objects $A$ and $B$ in the category $\textbf{C}$ is an triple $(A \times B,
\pi_{1}, \pi_{2})$ where for all other objects $C$ in $\textbf{C}$ with
projections $f: C \rightarrow A$ and $g: C \rightarrow B$ the unique arrow
$\langle f, g\rangle : C \rightarrow A \times B$ can be formed such that the
following diagram commutes:

\[\begin{tikzcd}[sep=huge]
 & C \arrow[ld, "f"'] \arrow[rd, "g"] \arrow[d, "{\langle f, g\rangle}" description, dashed] &  \\
A & A \times B \arrow[l, "\pi_2"] \arrow[r, "\pi_1"'] & B
\end{tikzcd}\]

where the dashed arrow indicates uniqueness. This can be extended to
\textit{n}-ary products in the obvious way.

As is common with universal constructions, products (via the unique arrow
$\langle f, g \rangle$) are unique up to unique isomorphism.

Examples of products within familiar categories include the cartesian product
$\times$ in \textbf{Set}, defined as the set of all tuples of elements from two
separate sets.

If products can be formed for every finite set of objects in a category it is
said to be cartesian.

\section{Terminal Objects}
Terminal objects are constructions that capture the
minimal structure required to be an object within a category. They often
correspond to the trivial examples of certain constructions.

A terminal object of a category \textbf{C} is an object, T, such that, for all
other objects, $A$ in the category, there exists a unique arrow $!_{A}: A
\rightarrow T$.

As a commutative diagram:

\begin{tikzcd}[sep=huge]
A \arrow[d, "!_A", dashed] \\
T
\end{tikzcd}

As with products, terminal objects in categories are unique up to unique
isomorphism.

Examples of terminal objects in common categories include any singleton in
\textbf{Set} and the one element group in \textbf{Group}

\section{Exponential Objects}
\section{Limits/Colimits}
Limts and colimits are an abstract way of looking at universal constructions.

A limit/colimit captures the notion that, for most universal constructions, a
particular canonical object that satisfies a certain diagram is desired. The
canonicity of this object is obtained by formulating the object as an initial
or terminal object in some category of objects that satisfy the diagram.

A treatment of limits will be provided with colimits as the dual notion.

For limits, the category of objects that satisfy the diagram is the category of
cones to a diagram.

A diagram is a functor from some indexing category (most often finite) that allows
us to pattern the category we are working with.

A cone to a diagram, $F: J \rightarrow C$, is an object $A: C$ such that for
all objects, $Y$, in $J$ there exists an arrow $\phi_{Y}$ such that the
following diagram commutes:

\[\begin{tikzcd}[sep=huge]
     & A \arrow[ld, "\phi_X"'] \arrow[rd, "\phi_Y"] &  \\
     F(X) \arrow[rr, "F(f)"] &  & F(Y)
 \end{tikzcd}\]

For some indexing category, \textbf{J}, the category of cones to a diagram can be
considered. The objects in this category are objects in the underlying category,
\textbf{C}, with a family of morphisms $\phi_{X}$ for all objects $X, Y: J$ in the indexing
category. Morphisms in this category are arrows between objects in \textbf{C}
such that the following diagram commutes.

\[\begin{tikzcd}[sep=huge]
     & A \arrow[ldd, "\psi_{X}"', bend right] \arrow[rdd, "\psi_Y", bend left]
     \arrow[d, "g" description] &  \\
      & B \arrow[ld, "\phi_X"'] \arrow[rd, "\phi_Y"] &  \\
      F(X) \arrow[rr, "F(f)" description] &  & F(Y)
\end{tikzcd}\]

A limit to a diagram is a terminal object in the category of cones to the
diagram.

The application of this construction can already be seen if the indexing
category is taken to be a discrete category (a category with only objects and
no morphisms). In this case, a cone to a functor from a discrete category will
appear as the bottom half of the diagram for products. Taking the terminal
object in the category of cones to this functor becomes the whole product
diagram.

If we take our indexing category to be the empty category, (i.e. the category
with no objects). The limit of this diagram is the terminal object.
\section{Universal Properties}
As is the norm in category theory, if a common pattern is found the aim is to
abstract it to find its underlying mechanism. Universal properties can
be abstracted as initial or terminal morphisms.

\subsection{Initial Morphisms}
An initial morphism is an initial object in the category $X \downarrow U$ where
$U : C \rightarrow D$ is a functor and $X$ is an object in $C$. More precisely,
an initial morphism is a pair $(A, \varphi)$ such that the following diagram
commutes.

A terminal morphism is a terminal object in the category $U \downarrow X$

\[\begin{tikzcd}
    A \arrow[rd] \arrow[r, "\phi"] & B \\
                                   & C
\end{tikzcd}\]

\section{Adjunctions}
\section{Yoneda Lemma and Embedding}
The Yoneda lemma is a core result in Category Theory. In a liberal sense it can
be thought to justify that by examining the ways in which an object can be
manipulated we can totally determine the object. This is a consequence of the
Yoneda Embedding, a functor from a (locally small) category $C$ to the category of
\textit{presheaves} (functors of type : $C \rightarrow \textbf{Set}$) on $C$.
The Yoneda lemma guarantees that this functor is fully faithful and injective on
objects, allowing the category of presheaves to be worked with in place of the
category itself. The complexity of this theorem in comparison to other
categorical constructs motivates a full treatment below.

Concretely the Yoneda lemma is the following statement:

\begin{equation}
    Nat(h^{A}, F) \cong FA
    \label{Yoneda}
\end{equation}

and that this isomorphism is natural in both $A$ and $F$.

The left hand side of \ref{Yoneda} can be understood as follows. $\phi : Nat(h^{A},
F)$ are natural transformations between the covariant hom-functor and some other
functor $F$, both presheaves on some underlying category $C$. Components of
$\phi$ at $X$ take morphisms from $A$ to $X$ to elements of the set $FX$.

Naturality will be dealt with after first establishing the isomorphism in
\ref{Yoneda}.

To establish the isomorphism, two functions must be defined. $Y: Nat(h^{A}, F)
\rightarrow FA$ and $Y^{op}: FA \rightarrow Nat(h^{A}, F)$ that are mutual
inverses.

The two functions are (using $\lambda$ abstraction for convenience):

% Y(\phi) = \phi_{A}(id_{A})
% Y^{op}(u) = \lambda f . F f u
\begin{align*}
    Y(\phi) = \phi_{A}(id_{A}) \\
    Y^{op}(u) = \lambda f . F f u
\end{align*}

$Y \circ Y^{op}$ can be shown to be $id$ as follows:

\begin{align*}
    (Y \circ Y^{op}) \ u &= Y (\lambda f \,.\, F f \ u) \\
    &= (\lambda f \,.\, F f \: u ) \ id_{A} \\
    &= (F \ id_{A}) \ u \\
    &=  id_{FA} u \\
    &= u
\end{align*}

To show $Y^{op} \circ Y \ = \ id$ proceed with:

\begin{align*}
    (Y^{op} \circ Y) (\phi) &= Y^{op} (\phi_{A}(id_{A})) \\
    &= \lambda f \, . \, F f \ (\phi_{A}( id_{A}))) \\
    &= \lambda f \, . \, \phi_{Y}(f) \\
    &= \phi
\end{align*}


Where the second to third line come from the fact that $\phi$ is a natural
transformation such that the following diagram commutes.

\[\begin{tikzcd}[sep=huge]
    {Hom(A, X)} \arrow[d, "{Hom(A,f)}"'] \arrow[rr, "\phi_{X}"] &  & FX
    \arrow[d, "Ff"] \\
    {Hom(A,Y)} \arrow[rr, "\phi_{Y}"] &  & FY
\end{tikzcd}\]

Now it must be shown that $Y^{op}(u)$ is a natural transformation i.e for some
$g: A \rightarrow Z$: 

\begin{align*}
    \lambda f \, . \, F f u  \circ hom(A, g) = F g \circ \lambda f \, . \, F f u
\end{align*}

Taking the left hand side on some $h: A \rightarrow X$:

\begin{align*}
    (\lambda f \, . \, F f u  \circ hom(A, g)) \ h &= (\lambda f \, . \, F f u) \ (
    hom(A, g) \ h) \\
    &= (\lambda f \, . \, F f u) \ (g \circ h) \\
    &=  F \ (g \circ h) \ u
\end{align*}

And now the right hand side:

\begin{align*}
    (F g \circ \lambda f \, . \, F f u) \ h &= F \ g \ ( (\lambda f \, . \, F f
        u) \
    h) \\
    &= F \ g \ (F \, h \, u) \\
    &= F \ (g \circ h) \ u
\end{align*}

Thus establishing $Y^{op}(u)$ as a natural transformation.

Finally naturality in $A$ and $F$ in \ref{Yoneda} must be established.

To establish naturality in $A$ it must be shown that, for all $f: C \rightarrow D$, the
following diagram commutes:

\[\begin{tikzcd}[sep=huge]
    {hom(h^C, F)} \arrow[d, "{hom(h^f, F)}"'] \arrow[rr, "Y"] &  & {F \,C}
    \arrow[d, "{F \, f}"] \\
    {hom(h^D, F)} \arrow[rr, "Y"] &  & {F \, D}
\end{tikzcd}\]

i.e.

\begin{align*}
    F \, f \circ Y = Y \circ hom(h^{f}, F)
\end{align*}

Beginning with the left hand side

\begin{align*}
    F \, f \ ( Y(\phi)) &= F \, f \ (\phi \ (id_{C})) \\
    &= (F \, f \circ \phi) \ (id_{C}) \\
    &= (\phi \circ h^{A}(f)) \ (id_{C}) \\
    &= \phi \ ((h^{A}(f) \ (id_{C})) \\
    &= \phi \ (f \circ id_{C}) \\
    &= \phi \, f
\end{align*}

And now the right

\begin{align*}
    (Y \circ hom(h^{f}, F)) \ (\phi) &= Y \ (hom(h^{f}, F) \, (\phi)) \\
    &= Y \, (\phi \circ h^{f}) \\
    &= (\phi \circ h^{f}) \ (id_{C}) \\
    &= \phi \ (h^{f} \, (id_{C})) \\ 
    &=  \phi \, f
\end{align*}

A contravariant version of Yoneda can be proven which states the following:

\begin{equation}
    Nat(h_{A}, F) \cong FA
    \label{coyoneda}
\end{equation}

$F$ can be fixed to certain functors to examine what Yoneda lemma means in
certain contexts.

Using the contravariant Yoneda and setting $F$ to be $h_{B}$ the equation

\begin{equation}
    Nat(h_{A}, h_{B}) \cong hom(A,B)
    \label{yonembedding}
\end{equation}


An obvious functor from any category \textbf{C} to the categor of presheafs from
\textbf{C} assigns the functor $h^{A}$ for all objects $A:\textbf{C}$ and
assigns the natural transformation $h^{f}$ to all morphisms $f$ in \textbf{C}.

\ref{yonembedding} states precisely that the image of the above described
functor on morphisms is bijective showing that the functor is fully faithful. It
is for this reason that is is known as the Yoneda embedding. The functor is also
injective on objects given that, by convention, hom-sets are disjoint\todo{check}. In this way, the Yoneda embedding allows a category \textbf{C} to
be examined by its embedding in the category of presheafs. We also have that two
objects are equal if the set of morphisms into and out of the objects are equal.
This justifies the statement that objects can be completely understood by the
way they relate to other objects.

Examples of the utility of Yoneda abound. Good illustrations can be found in
\todo{WhatAWhat you need to know about Yoneda}. As is a theme in Category
theory, Yoneda can be used to transport a problem into a simpler world.
\end{document}




