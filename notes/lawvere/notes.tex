\documentclass[a4paper,10pt]{article}

\usepackage{listings}
\usepackage[utf8]{inputenc}

\usepackage{amsmath}
\usepackage[margin=1.00in]{geometry}
\usepackage{cite}
\usepackage{titling}
\usepackage{indentfirst}
\usepackage[textwidth=3.7cm]{todonotes}
\setlength{\marginparwidth}{3.7cm}
\usepackage{etoolbox}
\lstset{breaklines}

\usepackage[compact]{titlesec}         % you need this package
\titlespacing{\section}{0pt}{0pt}{0pt} % this reduces space between (sub)sections to 0pt, for example
\titlespacing{\subsection}{0pt}{0pt}{0pt} % this reduces space between (sub)sections to 0pt, for example
\titlespacing{\subsubsection}{0pt}{0pt}{0pt} % this reduces space between (sub)sections to 0pt, for example
\AtBeginDocument{%                     % this will reduce spaces between parts (above and below) of texts within a (sub)section to 0pt, for example - like between an 'eqnarray' and text
    \setlength\abovedisplayskip{0pt}
    \setlength\belowdisplayskip{0pt}
}

\setlength{\droptitle}{-5em} % This is your set scream
\setlength{\parindent}{1em}
\setlength{\parskip}{0.3em}
\setlength{\parindent}{0in}

\date{}

\title{Diagonal Arguments and Cartesian Closed Categories\vspace{-20mm}}

\begin{document}
\maketitle
\section{Test}

A morphism, $f: A \rightarrow B$ is point surjective if for all points
$a: 1 \rightarrow A$ there exists a point $b: 1 \rightarrow B$ such that
$f \circ a = b$. This is not quite onto as there exists some categories that
have to much structure to spread all points i.e. categories with zero objects.

A fixed point of a morphism $f: B \rightarrow B$ is a point $s : 1 \rightarrow
B$ such that $f \circ s = s$.

In a cartesian closed category if there exists an object
$A$ and a point-surjective morphism $\phi: A \rightarrow B^{A}$ then every
endomorphism $f: B \rightarrow B$ has a fixed point.

For any $f$ we construct our fixed point $s$ as follows:

We must first consider the natural isomorphism induced by the tensor-hom
adjunction.

\begin{align*}
    hom(A\times B, Z) \cong hom(A, B^Z)
\end{align*}

Setting $A$ to be $\textbf{1}$ yields the isomorphism between morphisms from
$B \rightarrow Z$ and points to $B^{Z}$.

The fixed point construction works by precomposing our endomorphism $f: B
\rightarrow B$ with a function from $A \rightarrow B$ (We will call this
composition $g$). This means, from the point surjective $\phi$, that we can find
a point to $A$ associated with this arrow i.e. a $u: \textbf{1} \rightarrow A$
such that $\phi \circ u = f \circ g$.

If we now consider \textit{recomposing} $u$ to $g$ we get a point to $B$ we can
posit this object to be our fixed point. By writing out the equation that must
hold we can attempt to work out the necessary structure of $g$.

\begin{align*}
    &f \circ (g \circ u) = g \circ u \\
    &(f \circ g) \circ u = g \circ u \\
    &\phi \circ u \circ u = g \circ u \\
    &f \circ (\phi \circ u \circ u) = g \circ u
\end{align*}

Taking the last of these equations we can see for $g \circ u$ to be a fixed point it needs to take a point,
duplicate it, and apply both to the point surjective $\phi$ followed by
$f$. The cartesian
closedness of the category allows an arrow that does this to be constructed
directly.

Duplication can be done using $\delta(a): A \rightarrow A \times A \ = (a, a)$
guaranteed by the adjoint pair $\Delta  \dashv \times$.

for some $h: A \times A \rightarrow B$
\begin{align*}
    g \circ u &= h \circ \delta \circ u \\
    &= h \circ \langle u, u \rangle
\end{align*}

$h$ must take this pair of points and postcompose $f \circ \phi$ with each of them
individually. The first $u$ can be composed with $ f \circ \phi $ through the fusion rule
for composing products of morphisms.

\begin{align*}
    \langle f \circ \phi, id_A \rangle \circ \langle u, u \rangle &=
    \langle f \circ \phi \circ u, u \rangle \\
\end{align*}

Having composed the first $u$ with $\phi$, only the $u$ in the second projection
remains. Is is here that the adjoint property of the exponential is used by
utilising the evaluation map $eval: B^{A} \times A \rightarrow B$. Postcomposing
by the evaluation map yields a point to $B$ i.e.

\begin{align*}
    eval \ \circ \langle f \circ \phi \circ u, u \rangle
    &= eval \circ \langle f \circ \phi \circ u, u \rangle \\
    &= f \circ \phi \circ u \circ u
\end{align*}


\todo{is there a law here}

Unrolling what was just done yields
\begin{align*}
    g = eval \circ \langle f \circ \phi, id_A \rangle \circ \delta
\end{align*}

The existence of a fixed point combinator can be guaranteed by working within a
cartesian closed category where there is an iso $\psi: A \rightarrow B^A$
(although a split epi $D \rightarrow D^D$ will suffice) \todo{cite} 

\end{document}


