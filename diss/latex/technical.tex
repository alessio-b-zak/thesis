% !TEX root = ./dissertation.tex
This section will outline the essential technical details to understand the
primary contributions of this thesis. This section will aim to to provide a
working knowledge of the theory behind and the usage of the Agda theorem prover.
Through Agda, the underlying theory of categories will be explained to
sufficiently understand Lawvere's fixed point theorem. As explained prior,
theorem provers often work by utilising the Curry-howard isomorphism to embed a
logical framework within the type system of a programming language. This is the
approach taken by the Agda theorem prover which makes use of a type system
similar to that of Martin Lof Type theory, a type theory which provides a
logical framework for intutionistic higher-order logic.

\section{Martin Lof Type Theory}
The approach to theorem proving taken in this thesis is a type theoretic
approach exploiting the curry-howard correspondence. This will be done in a
dependent type theory designed by Per Martin-Lof, a logician working in
intutionistic logic. Martin Lof's type theory (MLTT) was intended to be an
entirely constructive foundation in which mathematics could be done. Just as,
inline with the curry-howard correspondence, the intutionistic fragment of
natural deduction has an interpretation within the simply-typed
$\lambda$-calculus, MLTT has a logical interpretation as first-order
intuitionistic predicate logic by including dependent types in the theory. The
theory is outlined below, a more detailed examination can be found notes derived
from Martin-Lof's lectures \cite{martin1984intuitionistic} or chapter one of the
\textit{Homotopy Type Theory} book \cite{hottbook}. Martin-Lof's lecture notes
discuss the philosophical ramifications of his theory and advocate for its use
as a foundation for all of mathematics.

MLTT adopts the Church-style view of typing in which terms cannot be brought in
without a unique type. This is done by introducing typing as a judgement of the
type theory. The judgements of a theory are the conclusions that can be made
given the deductive rules of said theory. The judgement that a term $a$ has type
$A$ is written
\begin{align*}
    a: \, A
\end{align*}

The other primary judgement of MLTT is that of judgemental equality. Judgemental
equality is used to introduce new constructions into the theory as definitional
equalities to allow us to introduce and name new constructions and to establish
when two terms normalise to the same term. The judgement that two terms,
$a$ and $b$ are equal (at type $A$) is written

\begin{align*}
    a \equiv b \, : \, A
\end{align*}

Types are introduced by introducing definitional equalities that define how to
form and use the types within the context of the theory. In the section that
follows, only a limited number of types will be introduced so as to understand
how the type theory can be used in theorem proving. Among the most basic
components of MLTT is the dependent function or Pi type, $\Pi$. $\Pi$ types
represent functions where the output type of a function can depend on the
argument to the function.

(dependent) Function types: intro, formation, computation, elimintation, propeq
sigma: intro, formation, computation, elimintation, propeq
product: intro, formation, computation, elimintation, propeq

propositions as types

Universes/ predicativity

\section{Agda}

Agda is a dependently typed functional language in which theorem can be done.
Agda can be used as a proof assistant through the lens of the curry-howard
correspondance given it's type system. Agda uses Haskell like syntax and its
underlying type theory is based of that of Per Martin-Lof. The following
describes features of Agda available in version 2.52, the version employed in
this thesis. Agda uses a predicative hierarchy of universes,
\AgdaPrimitive{Set}, indexed by a natural-number like type
\AgdaPrimitive{level}. \AgdaPrimitive{Set} can be viewed as a type-family
returning a universe.

A term \verb|t| of a given type, \AgdaDatatype{A} can be introduced and named
in Agda as follows

\ExecuteMetaData[../agda/latex/ThesisMisc.tex]{misc-name}


where \AgdaDatatype{name} : \AgdaDatatype{A} indicates that the identifier
\AgdaDatatype{name} has type \AgdaDatatype{A} and the second line assigns to
\AgdaDatatype{name} the value to the right hand sign of the equals sign,
\AgdaDatatype{t}. Unlike other typed programming languages, the types of
identifiers cannot be elided and inferred by the compiler. In the presence of
dependent types, the problem of type inference becomes undecidable and therefore
it is preferable to explicitly annotate them for all identifiers. Agda provides
a limited number of the constructions provided in Martin Lof type theory and
instead provides a method for defining inductive datatypes allowing inductive
types such as dependent pairs, product and coproduct types to be defined. One of
the constructions provided by are Pi types. A Pi type can be introduced with an
identifier, \AgdaFunction{foo}, as follows

\ExecuteMetaData[../agda/latex/ThesisMisc.tex]{misc-pi}

where, in the type signature, \verb|x| is an identifier of type \AgdaDatatype{A}
which may appear in the output type \AgdaDatatype{B}. The \verb|x| to the
lefthand side of the equals sign in the function definition binds the identifier
\verb|x| as an argument to the function of type \AgdaDatatype{A} which may be
used in the term \AgdaDatatype{M} of type \AgdaDatatype{B}. Arguments to a
function need not have the same names in the type and definition of the function
e.g.
\ExecuteMetaData[../agda/latex/ThesisMisc.tex]{misc-pi'}

is a valid definition.

As in MLTT, functions can also be introduced using a $\lambda$-abstraction. A
downside of this is that the arguments of a $\lambda$-abstraction can not be
pattern matched on in the body of a function.


Inductive data types can be introduced as follows.

\ExecuteMetaData[../agda/latex/ThesisMisc.tex]{misc-inductive1}

The \AgdaKeyword{data} keyword is followed by the name of the inductive data
type. Before the colon in the first line are the parameters to the type.
Parameters to a type appear as is in constructors to the type. They indicate
that the type behaves parametrically with respect to them. This is why in the
constructors \verb|Parameter| appears as is. Indices, on the other hand, appear
after the colon. Indices can change the shape of the type depending on the
constructor. In the example given the index is an inhabitant of the natural
number type and the first constructor dictates the inhabitant of the natural
number type is 0 and in the second constructor that the inhabitant is 1. The
final term before the \AgdaKeyword{where} statement is the output univerese of
the type i.e. to which \AgdaDatatype{Set} the type belongs. The constructors of
the Inductive data type are the possible inhabitants of the type. When pattern
matching onto an inductive data type, information is gained about the type based
on the index corresponding to the constructors produced by the pattern match.

Many constructions within Agda are common enough that they are desirable at all
levels of universe. Agda does not have cumulativity so, to assist universe
generic programming, universe polymorphism can be used.

A more useful example to consider is the sized list type, or vector

\ExecuteMetaData[../agda/latex/ThesisMisc.tex]{misc-inductive}

The curly braces in the type definition indicate implicit arguments, arguments
Agda will try and infer from other arguments. \AgdaDatatype{Vec} features set
polymorphism, parameters and indices. The two constructors for
\AgdaDatatype{Vec}, are indexed by their size. The empty list has size
\AgdaInductiveConstructor{zero} to indicate its emptiness.
\AgdaOperator{\AgdaInductiveConstructor{\AgdaUnderscore{}∷\AgdaUnderscore{}}}
takes an element of type \verb|a| and a Vector of size \verb|n| and appends the
singleton to the beginning creating a vector of size
\AgdaInductiveConstructor{suc} \verb|n|. When pattern matching onto a
\AgdaDatatype{Vec}, information is gained about the inductive argument to the
type i.e for the empty list has size \AgdaInductiveConstructor{zero} and in the
inductive case that the list has size \AgdaInductiveConstructor{suc} \verb|n|
for some \verb|n|. The remaining important types of Martin-Lof type theory can
now be introduced as inductive types beginning with Sigma types.
\todo{mixfix syntax}

\ExecuteMetaData[../agda/latex/ThesisMisc.tex]{misc-sigma}

The output universe for sigma must be the maximum of the levels of the first and
second components of the sigma type which can be done using \todo{lub}. The
product type is defined similarly.

\ExecuteMetaData[../agda/latex/ThesisMisc.tex]{misc-product}

and coproducts

\ExecuteMetaData[../agda/latex/ThesisMisc.tex]{misc-coproduct}

Instead of introducing induction and recursion principles for inductive data
types, Agda instead opts for deep pattern matching whereby an inductive datatype
can be expanded into its constitutent components. This is justified by the fact
that all inductive data types possible of being defined within (normal) Agda
correspond to W types i.e. types that admit a well-founded induction principle.
An example of pattern matching can be used to define a projection out of product
types.

\ExecuteMetaData[../agda/latex/ThesisMisc.tex]{misc-projl}

\todo{pattern match that introduces information}

The final type needed before propositions-as-types can be employed is the
equality type which is defined as follows

\ExecuteMetaData[../agda/latex/ThesisMisc.tex]{misc-equality}

For each inhabitant of \verb|A|, \verb|x|, there is unique inhabitant of the
equality type parameterised by \verb|x| which is when the second indexed
argument to the equality type normalises to the first parameter. This
restriction can only be made if the second argument is an index so we are able
to restrict its shape. Pattern matching on an equality proof in a function
definition released evidence into the environment that the two arguments are in
fact the same. This can be used in the body of the function as needed.

Proving theorems with no additional features in the language would prove
difficult. Mathematical structures would be a pain to define in the standard
inductive style as often they will often consist of a single constructor with
arguments that depend on each other. To assist with these types of structure
record types exist. Records are extensions of $\Sigma$ types which have named
fields to assist with referring to the individual components, an illustrative
example is the definition of a monoid, $(S \, , \, \bullet , \, e)$, within Agda

\ExecuteMetaData[../agda/latex/ThesisMisc.tex]{misc-monoid}

where \AgdaDatatype{lsuc} is the equivalent of \AgdaDatatype{suc} but for
levels. \todo{underscore}This is required due to \AgdaField{$\bullet$} which
forces the implicit constructor for the type to be of a higher sort than
\AgdaField{S} itself, due to the predicativity of the underlying type theory.
The identifiers to the left hand side of the colon under the \AgdaKeyword{field}
keyword in the above definitions define projections out of a
\AgdaDatatype{Monoid} object. The first three fields correspond to the elements
of a tuple representing a monoid, the underlying set, binary operation and
identity element. Unlike definitions based in set theory, the axioms the
structure must adhere to are considered to be fields in the object, a property
enabled by the proof-relevance of the type theory. Before categories are
introduced in mathematics and Agda, one more point must be addressed.

\subsection{Setoids}
The problem, being addressed can be introduced by considering a homomorphism
type for the above definition of monoid

\begin{AgdaMultiCode}
\ExecuteMetaData[../agda/latex/ThesisMisc.tex]{misc-monhom}
\ExecuteMetaData[../agda/latex/ThesisMisc.tex]{misc-monhom1}
\end{AgdaMultiCode}

where the fields of the second monoid are postfixed with an apostrophe. Consider
showing two monoids are propositionally equal. For records, this amounts to
showing that its fields are equal (due to $\eta$ for records?) i.e that the
underlying functions are the same but \textit{also} the proofs of preservation
of identity and and operation are the same. The proof-relevant nature of the
underlying type theory enables two different proofs to be distinguished by their
normal form. It would be unreasonable and beside-the-point to demand this when
equating two monoids but it is \textit{required} when equating using
propositional equality. Quotients and subsets plus extensionality.

This is a well understood problem in formalising
mathematics within type theories in theorem provers and there are several
solutions to this problem which will be addressed later.

A common solution within the type theory of unmodified Agda is the setoid
approach. The setoid approach 

observational type theory

quotients

setoids


\section{Category Theory}

