% !TEX root = ./dissertation.tex
\section{Agda-fying Categories}

\subsection{Categories}

Categories can be constructed in Agda with relative simplicity. The initial
concepts introduced here are taken from \verb|cats|, a Category Theory library
in Agda by Jannis Limpberg. \AgdaBound{lo}, \AgdaBound{la} and \AgdaBound{l≈}
are used in the following section as variables of type \AgdaDatatype{Level}.

Categories can be introduced as a record parameterised by the level of their
objects, arrows and type of morphism equality respectively
\ExecuteMetaData[../agda/latex/ThesisCategory.tex]{cat-def}

Categories are typed at a level above the largest of objects, arrows and
equalities in order to present equality on morphisms as relations on types as
per the Agda standard library. The axioms of a category are presented as fields
of the record. Objects are permitted to be any type of any level and mophisms
typed as any function that takes two objects and returns a type.

Identity arrows and composition take their obvious definitions. The identity
arrow provides a distinguished morphsim for each object implicitly and
composition is a function that takes two morphisms of the correct shape and
returns the appropriate morphism
\ExecuteMetaData[../agda/latex/ThesisCategory.tex]{cat-field-comp-id}

To codify the associativity of composition and the neutrality of identity it is
not practical to use propositional equality. Proving these as propositional
equalities of the structures that are often categorified is often a difficult
task. Propositional equality for a structure amounts to showing propositional
equality on its constituent components. That two terms are propositionally equal
implies the terms have the same normal form. There are often situations where it
is desirable to equate two different terms that do not have the same normal
forms. As an example, consider equating propositionally two monoid
homomorphisms. In a dependently typed setting this requires that the proofs of
preservation for the two morphisms are also equal in addition to the functions
between the underlying monoids. This is a tedious requirement that is often
undesired. \todo{Axiom K?}To remedy this, the earlier mentioned setoid model is
used whereby structures are equated by an equivalence relation.\todo{hott} For
categories an equivalence relation for morphism must be provided
\ExecuteMetaData[../agda/latex/ThesisCategory.tex]{cat-field-rel}

The \AgdaFunction{IsEquivalence} function establishes the appropriate proofs of
reflexivity, transitivity and symmetry.

With the notion of equality of morphisms in place it is possible to state the
properties of composition and identity

\ExecuteMetaData[../agda/latex/ThesisCategory.tex]{cat-field-comp}

The field \AgdaField{∘{-}resp} is a necessary and inconvenient aspect of
utilising equivalence relations over propositional equality. Preserves2
indicates that composition is congruent in both of its arguments. This allows us
to target individual compositions in a large categorical term to apply an
equality. This is given for free when using propositional equality as functions
are unable to distinguish terms with the same normal form. This can be seen as
one of the downsides to using equivalence relations as congruence must be proven
for every each individual equivalence relation.\todo{congruence}


\subsection{Uniqueness}

The next step in the categorical journey is a critical component of many of the
more complex abstractions, the notion of unique arrow

\ExecuteMetaData[../agda/latex/ThesisUnique.tex]{unique-def-unique}

Uniqueness is often given with respect to a property (hence universa
properties). In Agda this amounts to formualating the property as type
parameterised by the property and an arrow satisfying the property. The type
encodes a function which, given any other arrow satisfying the property
expresses equaltiy to the parameterised arrow. Using this a general unique arrow
can be encoded using a trivial function that always returns the unit.

Universal properties can now be given as an object, a proposition and a proof of
uniqueness

\ExecuteMetaData[../agda/latex/ThesisUnique.tex]{unique-def-exun}

Agda's syntax directive can be used to create a universal property like type
\todo{syntax}, as in the characterisation of an object as terminal whereby an
object is terminal if, for any other object in the category, there is an arrow
from the object to the terminal alongside a proof of uniqueness.

\ExecuteMetaData[../agda/latex/ThesisUnique.tex]{unique-def-terminal}

A category having a terminal object can now be encoded as a proposition which
takes a category and provides an object alongside a proof that it is terminal

\ExecuteMetaData[../agda/latex/ThesisTerminal.tex]{terminal-def-has}

To prove Lawvere's Fixed Point Theorem within Agda and the various surrounding
proofs in logic, Cartesian Closed Categories must be formalised. For this, in
addition to terminal objects, binary products and exponentials will be needed

\ExecuteMetaData[../agda/latex/ThesisCCC.tex]{ccc-def-is-ccc}

Products and exponentials have slightly more involved definitions. Beginning
with the uniqueness principle for products

\ExecuteMetaData[../agda/latex/ThesisProduct.tex]{product-def-product}

\AgdaFunction{IsProduct} takes an indexing function, a product object, and some
projections out of the product into the components of the indexing
category. \AgdaFunction{IsProduct} returns a functon type which, upon being
supplied a set of projections from an object to the indexing set, returns a
unique arrow from the the object to the previously supplied product object
satisfying the commuting diagrams for the product.


\subsection{Points}
All that follows is work done during this thesis. The theorem makes use of the
notion of points and some surrounding definitions

Points are a categorical abstraction that generalise the notion of elements of
a set. A point is an arrow from the terminal object to any other object

\ExecuteMetaData[../agda/latex/ThesisPoints.tex]{point-def-point}

Given the name of Lawvere's theorem it makes sense to formalise the notion of a
fixed point categorically

\ExecuteMetaData[../agda/latex/ThesisPoints.tex]{point-def-fixed-point}

A fixed point of a morphsism is a point that is idempotent under composition to
the right with the morphsim. As is common within the \verb|cats| library this is
then wrapped up in a record as a sigma type to express that a given function has
a fixed point

\ExecuteMetaData[../agda/latex/ThesisPoints.tex]{point-def-has-fixed-point}

Now the fixed point point property which features in Lawvere's theorem can be
formalised which is a predicate on a object in a category expressing that all
endomorphisms on the object have a fixed point.

\ExecuteMetaData[../agda/latex/ThesisPoints.tex]{point-def-fixed-point-property}





\paragraph{Example paragraph.}

This is an example paragraph; note the trailing full-stop in the title,
which is intended to ensure it does not run into the text.
