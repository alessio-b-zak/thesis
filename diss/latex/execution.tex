% !TEX root = ./dissertation.tex
\section{Agda-fying Categories}

\subsection{Categories}

Categories can be constructed in Agda with relative simplicity. The initial
concepts introduced here are taken from \verb|cats|, a Category Theory library
in Agda by Jannis Limpberg. \AgdaBound{lo}, \AgdaBound{la} and \AgdaBound{l≈}
are used in the following section as variables of type \AgdaDatatype{Level}.

Categories can be introduced as a record parameterised by the level of their
objects, arrows and type of morphism equality respectively
\ExecuteMetaData[../agda/latex/ThesisCategory.tex]{cat-def}

Categories are typed at a level above the largest of objects, arrows and
equalities in order to present equality on morphisms as relations on types as
per the Agda standard library. The axioms of a category are presented as fields
of the record. Objects are permitted to be any type of any level and mophisms
typed as any function that takes two objects and returns a type.

Identity arrows and composition take their obvious definitions. The identity
arrow provides a distinguished morphsim for each object implicitly and
composition is a function that takes two morphisms of the correct shape and
returns the appropriate morphism
\ExecuteMetaData[../agda/latex/ThesisCategory.tex]{cat-field-comp-id}

To codify the associativity of composition and the neutrality of identity it is
not practical to use propositional equality. Proving these as propositional
equalities of the structures that are often categorified is often a difficult
task. Propositional equality for a structure amounts to showing propositional
equality on its constituent components. That two terms are propositionally equal
implies the terms have the same normal form. There are often situations where it
is desirable to equate two different terms that do not have the same normal
forms. As an example, consider equating propositionally two monoid
homomorphisms. In a dependently typed setting this requires that the proofs of
preservation for the two morphisms are also equal in addition to the functions
between the underlying monoids. This is a tedious requirement that is often
undesired. \todo{Axiom K?}To remedy this, the earlier mentioned setoid model is
used whereby structures are equated by an equivalence relation.\todo{hott} For
categories an equivalence relation for morphism must be provided
\ExecuteMetaData[../agda/latex/ThesisCategory.tex]{cat-field-rel}

The \AgdaFunction{IsEquivalence} function provides the appropriate proofs of
reflexivity, transitivity and symmetry.

\begin{algorithm}[t]
\For{$i=0$ {\bf upto} $n$}{
  $t_i \leftarrow 0$\;
}
\caption{This is an example algorithm.}
\label{alg}
\end{algorithm}

\begin{lstlisting}[float={t},caption={This is an example listing.},label={lst},language=C]
for( i = 0; i < n; i++ ) {
  t[ i ] = 0;
}
\end{lstlisting}

This is an example sub-section;
the following content is auto-generated dummy text.
Notice the examples in Figure~\ref{fig}, Table~\ref{tab}, Algorithm~\ref{alg}
and Listing~\ref{lst}.
\lipsum

\subsubsection{Example Sub-sub-section}

This is an example sub-sub-section;
the following content is auto-generated dummy text.
\lipsum

\paragraph{Example paragraph.}

This is an example paragraph; note the trailing full-stop in the title,
which is intended to ensure it does not run into the text.
