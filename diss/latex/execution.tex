% !TEX root = ./dissertation.tex
\section{Agda-fying Categories}

\subsection{Categories}

Categories can be constructed in Agda with relative simplicity. The initial
concepts introduced here are taken from \verb|cats|, a Category Theory library
in Agda by Jannis Limpberg. \AgdaBound{lo}, \AgdaBound{la} and \AgdaBound{l≈}
are used in the following section as variables of type \AgdaDatatype{Level}.

Categories can be introduced as a record parameterised by the level of their
objects, arrows and type of morphism equality respectively
\ExecuteMetaData[../agda/latex/ThesisCategory.tex]{cat-def}

Categories are typed at a level above the largest of objects, arrows and
equalities in order to present equality on morphisms as relations on types as
per the Agda standard library. The axioms of a category are presented as fields
of the record. Objects are permitted to be any type of any level and mophisms
typed as any function that takes two objects and returns a type.

Identity arrows and composition take their obvious definitions. The identity
arrow provides a distinguished morphsim for each object implicitly and
composition is a function that takes two morphisms of the correct shape and
returns the appropriate morphism
\ExecuteMetaData[../agda/latex/ThesisCategory.tex]{cat-field-comp-id}

To codify the associativity of composition and the neutrality of identity it is
not practical to use propositional equality. Proving these as propositional
equalities of the structures that are often categorified is often a difficult
task. Propositional equality for a structure amounts to showing propositional
equality on its constituent components. That two terms are propositionally equal
implies the terms have the same normal form. There are often situations where it
is desirable to equate two different terms that do not have the same normal
forms. As an example, consider equating propositionally two monoid
homomorphisms. In a dependently typed setting this requires that the proofs of
preservation for the two morphisms are also equal in addition to the functions
between the underlying monoids. This is a tedious requirement that is often
undesired. \todo{Axiom K?}To remedy this, the earlier mentioned setoid model is
used whereby structures are equated by an equivalence relation.\todo{hott} For
categories an equivalence relation for morphism must be provided
\ExecuteMetaData[../agda/latex/ThesisCategory.tex]{cat-field-rel}

The \AgdaFunction{IsEquivalence} function establishes the appropriate proofs of
reflexivity, transitivity and symmetry.

With the notion of equality of morphisms in place it is possible to state the
properties of composition and identity

\ExecuteMetaData[../agda/latex/ThesisCategory.tex]{cat-field-comp}

The field \AgdaField{∘{-}resp} is a necessary and inconvenient aspect of
utilising equivalence relations over propositional equality. Preserves2
indicates that composition is congruent in both of its arguments. This allows us
to target individual compositions in a large categorical term to apply an
equality. This is given for free when using propositional equality as functions
are unable to distinguish terms with the same normal form. This can be seen as
one of the downsides to using equivalence relations as congruence must be proven
for every each individual equivalence relation.\todo{congruence}


\subsection{Uniqueness}

The next step in the categorical journey is a critical component of many of the
more complex abstractions, the notion of unique arrow

\ExecuteMetaData[../agda/latex/ThesisUnique.tex]{unique-def-unique}

Uniqueness is often given with respect to a property (hence universa
properties). In Agda this amounts to formualating the property as type
parameterised by the property and an arrow satisfying the property. The type
encodes a function which, given any other arrow satisfying the property
expresses equaltiy to the parameterised arrow. Using this a general unique arrow
can be encoded using a trivial function that always returns the unit.

Universal properties can now be given as an object, a proposition and a proof of
uniqueness

\ExecuteMetaData[../agda/latex/ThesisUnique.tex]{unique-def-exun}

Agda's syntax directive can be used to create a universal property like type
\todo{syntax}, as in the characterisation of an object as terminal whereby an
object is terminal if, for any other object in the category, there is an arrow
from the object to the terminal alongside a proof of uniqueness.

\ExecuteMetaData[../agda/latex/ThesisUnique.tex]{unique-def-terminal}

A Category 



This is an example sub-sub-section;
the following content is auto-generated dummy text.
\lipsum

\paragraph{Example paragraph.}

This is an example paragraph; note the trailing full-stop in the title,
which is intended to ensure it does not run into the text.
