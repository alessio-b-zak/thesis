% !TEX root = ./dissertation.tex

\section{Introduction}

This thesis is a small exploration into the interconnected and mysterious worlds
of mathematical logic and computation. Since the advent of the fields in the
early 20th Century, links have been found in unusual places and advances in one
has helped understand the other. There are two primary focal points of this
thesis, the unification of disparate areas in mathematical logic and computing
via category theory, an offshoot of abstract algebra, and the rigorous
formalisation of mathematics within modern theorem provers. More precisely, a
particular theorem within Category Theory will be proven within the theorem
prover Agda and its applications are explored with a new application provided in
the context of the untyped $\lambda$-calculus. To theorem in question that will
be explored within this thesis is Lawvere's fixed point theorem discovered by
William Lawvere in. \todo{date}Lawvere's fixed point theorem is a statement
within the context of cartesian closed categories, categories which play a
crucial role within the study of computation and logic. Lawvere's fixed point
theorem is a categorical abstraction of the familiar class of diagonal arguments
employed throughout computer science and mathematics. The decision to formalise
lawvere's fixed point theorem within a theorem prover is not incidental; many of
the theorems abstracted by Lawvere's theorem play an important role in the
problem of providing a formal foundation in which to do mathematics.

Many of the applications of lawvere's fixed point theorem are taken from
Yanofsky's 2003 review paper \ldots \todo{yanofsky}

Given the attempts of this thesis to incoporate a large amount of related yet
distinct fields together the contextual history behind the primary areas must be
outlined.

\section{Foundations of Mathematical Logic and Computation}

\subsection{Cantor's Theorem}

The story begins slightly before the advent of the 20th century with the German
Mathematician Georg Cantor. Cantor (1845 - 1918), is considered the father of
set theory with his proofs of the differing cardinalities of the real and
natural numbers, the theory of ordinal and transfinite induction for proving
properties of these. A major piece of work by Cantor was his eponymous theorem
that the cardinality of a set is strictly smaller than the cardinality of its
powerset. Cantor's proof of this theorem made use of a so-called diagonal
argument, the first of its kind which, repeated through time in different
fields, was abstracted by Lawvere to give his fixed point theorem in category
theory subsuming previous instances.
\todo{cantor's theorem}

\subsection{Russell's Paradox}
Cantor, in founding set theory as a field of study, introduced his Naive set
theory as a foundation in which to do mathematics, its naivety stemming from its
implicit Axiom of Unrestricted Comprehension, in which all predicates correspond
themselves to a set. This lead to perhaps the first signifcant result in the
foundations of mathematics in the 20th century, Russell's paradox. Betrand
Russell (1872-1970), an English polymath, showed that, in classical logic, the
axiom of unrestricted comprehension leads to a contradiction.

\todo{russell's paradox}

This result is relevenat to both prongs of this thesis. Russell's paradox
presented a blow to the hoped foundations of mathematics. Various systems were
designed to provide a more secure foundation of mathematics as a result of
Russell's paradox the type theories 

Russell's paradox
ZFC
\subsection{Hilbert's Program}
\subsubsection{Entsheidungsproblem}
\subsubsection{Other Problem}
\subsection{Computability}
\subsubsection{$\lambda$-calculus}
\subsection{Combinatory Logic}
\subsection{Curry Howard Iso}

\section{History of Theorem Proving}
\subsection{Principia Mathematica}
\subsection{Automath}
\subsection{Martin Lof Type Theory}
\subsection{Agda}
\subsection{Voevodsky}

\section{Category Theory and Mathematical Logic}
\subsection{Algebraic Topology}
\subsection{Lawvere's Functorial Semantics}
\subsection{Cartesian Closed Categories}
\subsection{Topoi}
\subsection{LFPT}
\subsection{Yanofsky}

