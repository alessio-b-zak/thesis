% !TEX root = ./dissertation.tex

\section{Introduction}

This thesis is a small exploration into the interconnected and mysterious worlds
of mathematical logic and computation. Since the advent of the fields in the
early 20th Century, links have been found in unusual places and both fields have
provided insight into the other. There are two primary focal points of this
thesis, the unification of disparate areas in mathematical logic and computing
via category theory, an offshoot of abstract algebra, and the rigorous
formalisation of mathematics within modern theorem provers. More precisely, a
particular theorem within category theory will be proven within the theorem
prover Agda and its applications are explored with a new application provided in
the context of the untyped $\lambda$-calculus. The theorem that will
be explored within this thesis is Lawvere's fixed point theorem discovered by
William Lawvere in 1969 in \textit{Diagonal Arguments and Cartesian Closed
Categories} \cite{lawvere1969diagonal}. \todo{date}Lawvere's fixed point theorem
is a statement within the context of cartesian closed categories, categories
which play a crucial role within the study of computation and logic. Lawvere's
fixed point theorem is a categorical abstraction of the familiar class of
diagonal arguments employed throughout computer science and mathematics. The
decision to formalise lawvere's fixed point theorem within a theorem prover is
not incidental; many of the theorems abstracted by Lawvere's theorem play an
important role in the problem of providing a formal foundation in which to do
mathematics.

The primary contributions of this thesis are: a formalisation of Lawvere's fixed
point theorem within the theorem prover Agda with additional proofs in the
theory of cartesian closed categories, a novel application of lawvere's fixed
point theorem within the context of the untyped $\lambda$-calculus with a
derivation of the first recursion theorem, and a formalisation within Agda of a
category of small types with presentations of Cantor's Theorem and Russell's
paradox within context.

Many of the applications of lawvere's fixed point theorem are taken from
Yanofsky's 2003 review paper \ldots \todo{yanofsky}.

Given the attempts of this thesis to incoporate a large amount of related yet
distinct fields together the contextual history behind the primary areas must be
outlined. What follows is a blurred and idealistic exposition of the core ideas
behind this thesis.

\section{Foundations of Mathematical Logic and Computation}

\subsection{Cantor's Theorem}

The story begins slightly before the advent of the 20th century with the German
Mathematician Georg Cantor. Cantor (1845 - 1918), is considered the father of
set theory with his proofs of the differing cardinalities of the real and
natural numbers, the theory of ordinals, and, for proving properties of
ordinals, transfinite induction. A major piece of work by Cantor was his
eponymous theorem \cite{cantor1892ueber}, published in 1892,
that the cardinality of a set is strictly smaller than the cardinality of its
powerset. Cantor's proof of this theorem made use of a so-called diagonal
argument, the first of its kind which, repeated through time in different
fields, was abstracted by Lawvere to give his fixed point theorem in category
theory subsuming previous instances.
\todo{cantor's theorem}

\subsection{Russell's Paradox}
Cantor, in founding set theory as a field of study, introduced his naive set
theory \cite{Cantor:1874} as a foundation in which to do mathematics, its
naivety stemming from its implicit axiom of unrestricted comprehension, in which
all predicates correspond themselves to a set. This lead to perhaps the first
signifcant result in the foundations of mathematics in the 20th century,
Russell's paradox. Betrand Russell (1872-1970), an English polymath, showed in
\textit{The Principles of Mathematics} written with Alfred Whitehead in 1903
that, in classical logic, the axiom of unrestricted comprehension leads to a
contradiction.

\todo{russell's paradox}

Rejecting the axiom of unrestricted comprehension prevented the theory of a
universal set from being considered.  This result is relevant to both prongs of
this thesis. Russell's paradox presented a blow to the hoped foundations of
mathematics. Various systems were designed to provide a more secure foundation
of mathematics as a result of Russell's paradox. Russell himself created the
first theories of types in his \textit{Principia Mathematica} \cite{russell25} to combat this
with various other foundational theories being presented, including set
theoretic foundations such as ZFC and NBG which reject the axiom of unrestricted
comprehension. As well as being intimately involved with the practicalities of
foundations itself, Russell's paradox can also be seen to be an instance of
Lawvere's theorem as outined in Yanofsky and presented formally further within
this thesis.

\todo{predicativity}

\subsection{Computability}
Shortly after the work of Cantor and Russell, the field of theoretical computer
science was birthed by attempts to answer the entsheidungsproblem (decision
problem), a challenge by David Hilbert \cite{hilbert1928theoretische} which required the formalisation of the
notion of algorithm.

\subsubsection{Entsheidungsproblem}\todo{recursive functions}

Devised in 1928 by Hilbert, the entsheidungsproblem asked for an effective
procedure or algorithm which, on an input of a statement in first-order logic
plus a finite numbers of axioms, could determine its validity in all structures
statisfying the axioms. Before an answer could be given, the notion of algorithm
had to be formalised. This was done in the 1930s independently by Alonzo Church
and Alan Turing. Turing's formalisation \cite{turing1937computable}, presented
in 1937, the class of Turing Machine's were machines aimed at directly capturing
the notion of algorithm, provided a negative answer to this question by
establishing the existence of undecidable problems such as his halting problem
and confirming the entscheidungsproblem as one of these undecidable problems.
Turing's proof of the existence of undecidable problems and his proof of the
Halting problem resemble in no small part the proof of Cantor's Theorem as
diagonalisation arguments. Yanofsky, shows how both proofs can be understood
through the lens of Lawvere's fixed point theorem constituting another
unification of a concept in mathematical logic by Lawvere's Theorem.
\subsubsection{$\lambda$-calculus}
Instead of capturing directly the notion of computability, Alonzo Church's
formalisation was focused on formalising more precisely the notion of function.
Church's formulation, the $\lambda$-calculus, is a formal system consisting of
the operations of abstraction and application using variable binding and
substitution. Church showed that the $\lambda$-calculus had an notion of
$\lambda$-computability which was equivalent to the class of general recursive
functions as defined by Godel and thus represented a model of computation.
Church also provided a negative answer to the entscheidungsproblem. The
$\lambda$-calculus has applications beyond that of providing a negative answer
to Hilbert's Problem. The $\lambda$-calculus originally arose form Church's
desire to provide a foundation for logic, a desire also shared by another
American Logician, Haskell Curry. In the 1920s and 1930s, Curry worked on a
foundation of logic extremely similar to the $\lambda$-calculus, his theory of
combinators. Combinatory logic, as defined by Curry, was in correspondence with
the $\lambda$-calculus. Curry observed in 1934 that, if types were assigned to
the domains and comdomains of his combinators, the types  matched the axioms of
intuitionistic implicational logic. This observation proved to be the beginning
of theorem provers as they are in their current state.

$\lambda$-calculus as a deductive system
Y-combinator and curry's paradox
the simply typed lambda calculus
lambda cube
curry howard

\subsection{Curry Howard Correspondence}
Curry's combinatory logic and Church's $\lambda$-calculus, when viewed as a
system for logic, proved to be inconsistent as shown by Kleene and Rosser and
again by Curry using his Paradoxical Combinator. To remedy this inconsistency
both Curry and Church extended their logical systems with types giving typed
combinatory logic and the simply typed $\lambda$-calculus. Further to Curry's
aformentioned observation made of typed combinatory logic, in the 1960s William
Alvin Howard  made a further observation that the the typing rules of the simply
$\lambda$-calculus corresponded to laws of inference for intuitionistic style
natural deduction. This lead to the conclusion that every system in
formal logic has a corresponding computational calculi with a specific type
system named the curry-howard correspondence.

Intuinistic logic and constructivity. bhk interpretation

\section{History of Theorem Proving}
\todo{define logical framework}
The curry-howard correspondence provided a computational intepretation of the
notion of proof. Providing a proof of a theorem corresponded to a program that
inhabited a given type. Type systems in computation had largely grown with the
theory of programming languages. Types were intended to check that a program, to
some degree, behaved as intended. It was understood that the comprehensive type
systems and their associated checkers could be used to check the validity of
proofs in mathematics. With the advent of physical computers this provided hope
that a more methodical approach to mathematics that eliminated the uncertainty
around new proofs that were published.
\subsection{Automath}
The first computational system to exploit the curry-howard isomorphism to act as
a theorem prover was Automath in 1967, slighty before Howard's explicit
observation of the curry-howard correspondence. Automath (automating
mathematics) was designed by Dutch Mathematician, Nicolaas Govert de Bruijni (of
index fame), who independently observed the curry-howard correspondence.
Automath was a typed programming language providing inbuilt support for variable
binding, substitution and application of judgemental equalities. As has been
a common feature of proof-assistants since and is the defining features of
computational calculi known as \textit{logical frameworks}. Users could
define their own logics and types and no method of introducing inductive types
was provided. Typechecking the program corresponded to the verification of the
proof. \todo{Proofs done with automath.}

\subsection{Martin Lof Type Theory}
In the early 1970s Per Martin-Lof, a Swedish logician and mathematician, aimed
to exploit the curry-howard correspondence to provide what he deemed as a better
foundation for mathematics. Per Martin-Lof, a constructivist, asserted that in
order to know of the existence of a mathematical object it must be directly
constructed. To this end Martin-Lof went about producing a type system to
produce an intuitionistic type theory for proving within higher order logics.
Intuitionistic logics were precisely the logics that type systems corresponded
to which matched Martin-Lof's constructivist agenda. Martin-Lof further aimed
for his type system to replace set theory as a foundation for doing all of
mathematics. Martin-Lof's type theory was incredibly successful and pioneered
the propositions-as-types approach to theorem proving which is an incredibly
common approach taken in modern theorem provers. An overview of his theory will
be presented in the technical background for this thesis.

Martin-Lof type theory was deeply influential in many theorem provers and
logical frameworks designed from then on including the Edinburgh Logical
Framework and proof-assistant Agda, the proof-assistant used within this thesis.

Martin-Lof's type theory was extended in future research to support different
domains and uses. Thierry Coquand, a French computer scientist, created the
calculus of constructions, a typed $\lambda$-calculus which he used to create
the proof-assistant Coq.

coq

NurP

Hol/Isabelle

Universes and hurkens

delinieating theorem provers and logical frameworks

\subsection{Agda}
The first version of Agda was designed at Chalmers university in 1999 by
Caterina Coquand based on the ALF logical framework ultimately derived from
Martin-Lof type theory. The second version of Agda was later designed by Ulf
Norell during his PhD also at Chalmers University in 2007 based on Zhaohui Luo's
\textit{Unified Theory of Dependent Types} (UTT) derived from Martin-Lof type
theory. Agda is implemented as a functional programming language with dependent
types, and features an interactive mode in the emacs text editor, allowing users
to interact with the Agda intepreter to develop proofs. Agda is a total
language, allowing the user to write only programs that are probably
terminating. This enables type checking (and therefore proof checking) to be
decidable. The standard Agda backend, \textit{MAlonzo}, is written in the
functional programming language Haskell and is still being extended and
developed in 2019.
\section{Category Theory and Mathematical Logic}
Alongside the developlment and exploration of  proof assistants, type theory and
the curry-howard correspondance the field of category theory was birthed and
matured. Category theory is an offshoot of abstract algebra and algebraic
topology and geometry that  aims to provide a general account of mathematical
structure. A category is a mathematical structure consisting of a collection of
objects and arrows between these objects with small set of further axioms to
which that the structure must adhere. This description allows many common
mathematical structures to be considered as a category such as the category of
sets where the objects are sets and the arrows functions or the category of
groups with the objects being groups and the arrows being group homomorphisms.
The definition of category is such that the majority of mathematical structures
can be considered as one. The abstract and encompassing definition of
categories provides a vehicle for the transposition of mathematical ideas from
one field to another. Category theory also provides a framework in which to
understand ubiqutous and reoccurring themes in mathematics such as free objects,
products and function spaces.
\subsection{Algebraic Topology}
Category theory was initially invented by American mathematicians Samuel
Eilenberg and Saunders Mac Lane in 1945 in their paper \textit{General Theory of
Natural Equivalences} which defined categories, structure-preserving maps
between categories or functors and structure preserving maps between functors or
natural transformations. Mac Lane and Eilenberg initially applied these to the
field of algebraic topology and geometry to make certain constructions simpler.
This line was followed by Grothendieck and Kan who added further concepts such
as adjoint functors and limits. A marked addition to the study of category
theory was made by one of Eilenberg's PhD students William Lawvere. Lawvere's
work throughout the 1960s began to analyse the relations between logic,
foundations of mathematics and category theory. Lawvere provided a categorical
and structural account of axiomatic set theory with his \textit{Elementary theory of the
Category of Sets}. His Ph.D. thesis explored model theory and introduced
Lawvere's theories, the categorical counterpart to equational theories.
\todo{elementary topoi}

\subsection{Diagonal Arguments and Cartesian Closed Categories}
Another staple in Lawvere's series of papers relating to category theory and
logic and a focal point of this thesis was his 1969 paper \textit{Diagonal
Arguments and Cartesian Closed Categories}. Cartesian closed categories are
a conceptual class of categories that have close relations to type theory and
logic. Centered around having an internal hom functor (internalising the concept
of a function space), cartesian closed categories arose from an analysis of
elementary topoi. Joachim Lambek in 1985 extended the curry-howard
correspondence by providing a correspondence between simply-typed lambda calculi
and cartesian closed categories establishing the religion of computational
trinitarianism. In \textit{Diagonal Arguments and Cartesian Closed Categories},
Lawvere showed how the paradoxes in mathematical logic and set theory from the
early 20th century could be unified under a single theorem in the theory of
cartesian closed categories. Lawvere's fixed point theorem, when intepreted
within the category of sets which is cartesian closed, produces Cantor's
theorem. Lawvere's theorem has the structure of a classical diagonal theorem and
has been used since to explain the structure and appearances of other paradoxes
and phenomena in mathematical logic. A comprehensive account of this is provided
in.

\todo{lidenbaum tarski algebras}

\subsection{Yanofsky}

\subsection{Contributions}
It has been noted that the components of the proof of Lawvere's fixed point
theorem resemble syntactically the construction of the curry's Y-combinator in
combinatory logic. Given the connection between cartesian closed categories and
the untyped lambda calculus and the syntactic resemblance it has been assumed by
some that there is a relation between the untyped lambda calculus and lawvere's
fixed point theorem. Given the theorems applications within the theory of
computable universes and Turing machines, recursion theory and the recursion
theorem and Rice's theorem it seems that this would be likely. In spite of this
the relation between the $\lambda$-calculus and Lawvere's fixed point theorem has
never been fully developed. The primary contribution of this thesis is an
explicit account of this relation, primarily that, when intepreted in the
context of cartesian closed categories corresponding to models of the untyped
$\lambda$-calculus, Lawvere's fixed point theorem is equivalent to the first
fixed point theorem for the untyped $\lambda$-calculus. In the evaluation of
this thesis, mistakes with previous attempts to formalise the relation are
provided. The proofs in this thesis are done in the proof-assistant, Agda. This
decision was made so as to explore the relation between type theory and
mathematical logic but has provided other benefits. Through formalisation, a
greater appreciation for the particularities of Lawvere's theorem can be
understood making it easier to find further relations. The other primary
contribution of this thesis is a collection of proofs within cartesian closed
categories and of Lawvere's theorem itself.

