\documentclass[ % the name of the author
                author={Alessio Zakaria},
                % the name of the supervisor
                supervisor={Dr. Nicolas Wu},
                % the degree programme
                degree={MEng},
                % the dissertation title (properly capitalised)
                title={Automated Theorem Proving in Category Theory and the
                $\lambda$-calculus},
                % the dissertation subtitle (which can be blank)
                subtitle={},
                % the dissertation type
                type={Research},
                % the year of submission
                year={2019} ]{dissertation}


\begin{document}

\listoftodos % Creates "Todo list" page at start of doc

% =============================================================================

% This macro creates the standard UoB title page by using information drawn
% from the document class (meaning it is vital you select the correct degree
% title and so on).

\maketitle

% After the title page (which is a special case in that it is not numbered)
% comes the front matter or preliminaries; this macro signals the start of
% such content, meaning the pages are numbered with Roman numerals.

\frontmatter

% This macro creates the standard UoB declaration; on the printed hard-copy,
% this must be physically signed by the author in the space indicated.
\makedecl
\todo{SIGN DECLARATION}

% LaTeX automatically generates a table of contents, plus associated lists
% of figures, tables and algorithms.  The former is a compulsory part of the
% dissertation, but if you do not require the latter they can be suppressed
% by simply commenting out the associated macro.

\tableofcontents

% The following sections are part of the front matter, but are not generated
% automatically by LaTeX; the use of \chapter* means they are not numbered.

% -----------------------------------------------------------------------------
\mainmatter

% -----------------------------------------------------------------------------
\chapter*{Acknowledgements}
Thank you to Jordan Peterson for being an inspiration
\chapter{Contextual Background}
\label{chap:context}



% !TEX root = ./dissertation.tex
{\bf A compulsory chapter,     of roughly $5$ pages}
\vspace{1cm}

\noindent
Theorem proving plus history. Category theory and applications to computing.
Diagonal arguments. lambda calculus.
Type theory and depedent types


automath, hilbert, russell type theory, martin lof, coq, agda, cedille, nurpl.
Intutionistic logic, bhk.

Maclane, lawvere, awodey, categorical logic, CCCs,

russell's paradox, church.


\begin{quote}
\noindent
The high-level objective of this project is to reduce the performance
gap between hardware and software implementations of modular arithmetic.
More specifically, the concrete aims are:

\begin{enumerate}
\item Research and survey literature on public-key cryptography and
      identify the state of the art in exponentiation algorithms.
\item Improve the state of the art algorithm so that it can be used
      in an effective and flexible way on constrained devices.
\item Implement a framework for describing exponentiation algorithms
      and populate it with suitable examples from the literature on
      an ARM7 platform.
\item Use the framework to perform a study of algorithm performance
      in terms of time and space, and show the proposed improvements
      are worthwhile.
\end{enumerate}
\end{quote}


% -----------------------------------------------------------------------------

\chapter{Technical Background}
\label{chap:technical}

% !TEX root = ./dissertation.tex
This section will outline the essential technical details to understand the
primary contributions of this thesis. This section will aim to to provide a
working knowledge of the theory behind and the usage of the Agda theorem prover.
Through Agda, the underlying theory of categories will be explained to
sufficiently understand Lawvere's fixed point theorem. As explained prior,
theorem provers often work by utilising the Curry-howard isomorphism to embed a
logical framework within the type system of a programming language. This is the
approach taken by the Agda theorem prover which makes use of a type system
similar to that of Martin Lof Type theory, a type theory which provides a
logical framework for intutionistic higher-order logic.

\section{\mlt{} Type Theory}
The approach to theorem proving taken in this thesis is a type theoretic
approach exploiting the \cuho{} correspondence. This will be done in the
dependent type theory designed by \mlt{}. Martin Lof's type theory (MLTT) was intended to be an
entirely constructive foundation in which mathematics could be done. Just as,
inline with the \cuho{} correspondence, the intutionistic fragment of
natural deduction has an interpretation within the simply-typed
$\lambda$-calculus, MLTT has a logical interpretation as first-order
intuitionistic predicate logic by including dependent types in the theory. The
theory is outlined below, a more detailed examination can be found notes derived
from \mlt's lectures \cite{martin1984intuitionistic} or Chapter One of the
\textit{Homotopy Type Theory} book \cite{hottbook}. \mlt's lecture notes
discuss the philosophical ramifications of his theory and advocate for its use
as a foundation for all of mathematics.

Within MLTT, typing takes the form of a judgement. Judgements are statements in
the metatheory of the type theory that can either be derived from the deductive
rules of the type theory or introduced independently. The judgement that a term
$a$ has type $A$ is written
\begin{align*}
    a: \, A
\end{align*}

The other primary judgement of MLTT is equality. Equality equates
two terms in the sense that one can be replaced with the other freely within the
theory. This allows new constructions to be introduced within the theory through
naming and functions. The judgement that
two terms, $a$ and $b$ are equal (at type $A$) is written

\begin{align*}
    a \equiv b \, : \, A
\end{align*}

Types are introduced by introducing equalities that define how to form and use
the types within the context of the theory. In the section that follows, only a
limited number of types will be introduced so as to understand how the type
theory can be used in theorem proving. Before defining types, however, what it
means to be a type must be introduced. A universe is a type whose elements are
themselves types. The version of MLTT presented here postulates an infinite
hierarchy of universes, $U_{n}$ which is an element of the universe at a higher
level $U_{n+1}$ i.e.

\begin{align*}
    U_{0} : U_{1} : U_{2} \cdots
\end{align*}

A type is then an inhabitant of some universe. One might desire some finite
number of universes in which to work and this was something \mlt{} desired.
Martin-Log's first presentation of his type theory featured only one universe,
$U$, which was an element of itself.

\begin{align*}
    U : U
\end{align*}

Thierry Coquand in 1992 showed that a Russell style paradox could be embedded
within a type theory with a single impredicative universe. This mirrors the
inconsistencies that develop when trying to postulate a set of all sets in naive
set theory. A type is then defined to bean element of one of the postulated
universes and new types can be defined within a given universe.



Introducing new types into the type theory involves explaining how to create
objects of that type and how to compute with objects of the type, done by adding
more definitional equalities to the system.  The most basic components of MLTT
is the dependent function or Pi-type, $\Pi$. $\Pi$-types represent functions
where the output type of a function can depend on the argument to the function.
If $A$ is a type (i.e. $A : U$ for some universe $U$) and $B : U$  the
$\Pi$-type,

\begin{align*}
    \prod_{(x : A)}B
\end{align*}
which binds the variable $x$ in B, represents the dependent function which takes
an argument $x : A$ and returns an element of type $B$ with the free variable
$x$ replaced with the argument to the function. A dependent function, $f :
\prod_{(x : A)}B$, can be introduced via a set of defining equations or via a
$\lambda$-abstraction. Given an expression $M : B$ where $B$ may contain the
variable $x : A$ $f$ can be defined as
\begin{align*}
    f (x) :\equiv M \textrm{ for } x : A
\end{align*}
Another way to introduce a dependent function is to introduce a $\lambda$ which
takes an identifier and an expression and produces a dependent function i.e.
\begin{align*}
    \lambda \, x . M : \prod_{(x : A)}B(x)
\end{align*}
Computing with $\prod$-types occurs through application and substitution. Given
a dependent function, $f(x) \equiv M : \prod_{(x : A)}B(x)$, and a value $a : A$, $a$ can be
applied to $f$ to obtain a value of type $B(a)$ written
\begin{align*}
    f(a) : B(a)
\end{align*}
\todo{uniqueness principle for pi types}

The expression that the the application yields is the result of replacing all
instances of $x$ in $M$ with $a$. With depenent functions it is possible to
capture the notion of a \textit{type-family}. A type-family is a $\Pi$-type that
returns an element of some universe written. The other types to be introduced in
this section can be written as inductive types. Inductive types can be
introduced by supplying constructors with build elements of the type from other
types.  Inductive types can be computed with using induction principles which
describe how to compute with abitrary structures of the type based on the
constituent parts. Induction principles can be seen as an alternative to pattern
matching, a more common feature of functional programming languages. A simple
type to illustrate this is the product type which is the type theory analogue of
the cartesian product. Given two types $A : U_{n}$ and $B : U_{n}$ the product
type, $A \times B : U$ can be formed. Given $a : A$ and $b : B$ an element of
the pair type $A \times B$ can be constructed as
\begin{align*}
    (a \, , \, \, b) \, : \, A \times B
\end{align*}
Before introducing the induction principle for product types it is worth
dwelling on some subtleties of induction principles. Induction principles are
judgemental equalities that are describe how to compute with newly introduced
data types. Induction principles often provide a separate defining function for
each constructor for a type however this is \textit{not} pattern matching. A
separate defining equation for each constructor often makes logical sense for
the type to have good computational properties but there is an element of choice
to designing induction principles as shall be seen in the discussion of the
identity type. The induction principle for products is a function with the type

\begin{align*}
    \textsf{ind}_{A \times B} : \prod_{C : A \times B\rightarrow
    U}(\prod_{(x : A)}\prod_{(y : B)}C((x , y))) \rightarrow \prod_{x : A\times
    B}C(x)
\end{align*}

Intuitively, this type can be read as, given a type family $C : A \times B
\rightarrow U$ for some universe, $U$, and given a dependent function which
takes two arguments returns the type family applied to the pair of the two
functions, a dependent function for pairs can be produced. More concisely, a
function for pairs can be produced from a function that takes two arguments.
Intuitively this can be done by taking the function with two arguments,
deconstructing the pair and applying each component of the pair in turn. As a
defining equation this is:
\begin{align*}
    \textsf{ind}_{A \times B}\, (C, g, (a , b)) :\equiv g(a)(b)
\end{align*}
Other types that are integral to the theorem-proving effort are $\Sigma$-types.
$\Sigma$-types or dependent pairs. Dependent pairs are product types where the
type of the second argument can depend on the first. Given a type $A : U$ and a
type-family $B : A \rightarrow U$ the dependent product type $\Sigma_{(x :
A)}B(x)$ can be formed. Given an element $a : A$ and an element $b : B(a)$ the
dependent pair $(a,b) : \Sigma_{(x : A)}B(x)$. The induction principle for
$\Sigma$-types is similar to the induction principle for products.

\begin{align*}
    \textsf{ind}_{\Sigma_{x :A}B(x)} : \prod_{C : \Sigma_{x : A }B(x)\rightarrow
    U}\bigg(\prod_{(x : A)}\prod_{(y : B)}C((x , y))\bigg) \rightarrow \prod_{p :
    \Sigma_{x : A}B(x)}C(x)
\end{align*}
with the same defining equation
\begin{align*}
    \textsf{ind}_{\Sigma_{x :A}B(x)}\, (C, g, (a , b)) :\equiv g(a)(b)
\end{align*}


The coproduct type is the typed variant of the disjoint union from set theory.
For two types $A : U$ and $B : U$ coproduct of $A$ and $B$ is $A + B : U$.
induction principle

unit
induction principle

void

A final and key component of MLTT is the identity type, a type used to prove
that two terms are equal. This is known as propositional equality and is
internal to the theory. For a given type, $A : U$, the identity type is a family
$\textsf{Id}_{A} : A \rightarrow A \rightarrow U$ for a given $a : A$, $b : A$
written $a =_{A} b$. Within MLTT there is a single inhabitant of the identity
type for a given  $a : A$  which can only be introduced if the second elements
of $A$ are definitionally equal known as $\textsf{refl}_{a}$. $\textsf{refl}$ is
a constructor of type
\begin{align*}
    \textsf{refl} : \prod_{a : A}(a =_{A} a)
\end{align*}

There are different choices for induction principle for the identity type.
Choosing whether to use both or only one is major decision in modern type
theories. One of the induction principles for the identity type is known as
\textsf{Axiom J}. \textsf{Axiom J} is a function of type

\begin{align*}
    \textsf{ind}_{=_{A}} : \prod_{(C : \Pi_{(x,y : A)}(x=_{A}y)\rightarrow
    U)}\bigg(\prod_{(x : A)}C(x,x,\textsf{refl}_{x})\bigg) \rightarrow
    \prod_{(x,y:A)}\prod_{p:x=_{A}y}C(x,y,p)
\end{align*}

\textsf{Axiom J} is defined as
\begin{align*}
    \textsf{ind}_{=_{A}} (C,c,x,x,\textsf{refl}_{x}) :\equiv c(x)
\end{align*}


The basis upon which MLTT can be used as a foundation for mathematics is via the
\cuho{} correspondence in a particular fashion known as
propositions-as-types.

\subsection{Propositions-as-types}
Propositions-as-types hinges on intepreting a proof of a proposition as an
inhabitant of a correponding type. The types presented in the previous section
are the types with which higher order intutionistic logic can be intepreted.
With types instead of sets, universal quantification over a type can be
simulated using $\Pi$-types where the output type of the dependent function is
the proposition being quantified over. Existential quantification can be
intepreted as $\Sigma$ types where the second argument is the proposition being
quantified over and the first argument is the object that satisfies said
proposition. Implication is a non-dependent function type, logical conjuction
corresponds to product types, coproduct types to logical disjunction. Truth is
inhabitation of the unit type and falsity as an inhabitant of the void type. The
void type has no inhabitants and so any such inhabitant would constitute a proof
of the inconsistency of the logic the type system represents. With this
definition of falsity, negation is a function that takes a type (proposition)
and produces and inhabitant of the void type, something that should not be
possible. The intutionistic, constructive component of this logic comes from the
fact that to prove a position an element of a type \textit{must} be constructed.
A proof of $A \land B$ consists of providing a proof of $A$ \textit{and}
providing a proof of $B$. This constructive approach to logic weakens the
deductive framework.

\todo{double negation/excluded middle}
The propositions-as-types intepretation of logic is
proof-relevant, proving, within the type theory, that two things are equal uses
the identity type. An inhabitant of the identity type is an object within the
type-theory presenting a equality between two things.


\section{Agda}

Agda is a dependently typed functional language in which theorem can be done.
Agda can be used as a proof assistant through the lens of the \cuho{}
correspondance given it's type system. Agda uses Haskell like syntax and its
underlying type theory is based of that of Per \mlt. The following
describes features of Agda available in version 2.52, the version employed in
this thesis. Agda uses a predicative hierarchy of universes,
\AgdaPrimitive{Set}, indexed by a natural-number like type
\AgdaPrimitive{level}. \AgdaPrimitive{Set} can be viewed as a type-family
returning a universe.

A term \verb|t| of a given type, \AgdaDatatype{A} can be introduced and named
in Agda as follows

\ExecuteMetaData[../agda/latex/ThesisMisc.tex]{misc-name}


where \AgdaDatatype{name} : \AgdaDatatype{A} indicates that the identifier
\AgdaDatatype{name} has type \AgdaDatatype{A} and the second line assigns to
\AgdaDatatype{name} the value to the right hand sign of the equals sign,
\AgdaDatatype{t}. Unlike other typed programming languages, the types of
identifiers cannot be elided and inferred by the compiler. In the presence of
dependent types, the problem of type inference becomes undecidable and therefore
it is preferable to explicitly annotate them for all identifiers. Agda provides
a limited number of the constructions provided in Martin Lof type theory and
instead provides a method for defining inductive datatypes allowing inductive
types such as dependent pairs, product and coproduct types to be defined. One of
the constructions provided by are Pi types. A Pi type can be introduced with an
identifier, \AgdaFunction{foo}, as follows

\ExecuteMetaData[../agda/latex/ThesisMisc.tex]{misc-pi}

where, in the type signature, \verb|x| is an identifier of type \AgdaDatatype{A}
which may appear in the output type \AgdaDatatype{B}. The \verb|x| to the
lefthand side of the equals sign in the function definition binds the identifier
\verb|x| as an argument to the function of type \AgdaDatatype{A} which may be
used in the term \AgdaDatatype{M} of type \AgdaDatatype{B}. Arguments to a
function need not have the same names in the type and definition of the function
e.g.
\ExecuteMetaData[../agda/latex/ThesisMisc.tex]{misc-pi'}

is a valid definition.

As in MLTT, functions can also be introduced using a $\lambda$-abstraction. A
downside of this is that the arguments of a $\lambda$-abstraction can not be
pattern matched on in the body of a function.


Inductive data types can be introduced as follows.

\ExecuteMetaData[../agda/latex/ThesisMisc.tex]{misc-inductive1}

The \AgdaKeyword{data} keyword is followed by the name of the inductive data
type. Before the colon in the first line are the parameters to the type.
Parameters to a type appear as is in constructors to the type. They indicate
that the type behaves parametrically with respect to them. This is why in the
constructors \verb|Parameter| appears as is. Indices, on the other hand, appear
after the colon. Indices can change the shape of the type depending on the
constructor. In the example given the index is an inhabitant of the natural
number type and the first constructor dictates the inhabitant of the natural
number type is \AgdaInductiveConstructor{0} and in the second constructor that
the inhabitant is \AgdaInductiveConstructor{1}. The final term before the
\AgdaKeyword{where} statement is the output univerese of the type i.e. to which
\AgdaDatatype{Set} the type belongs. The constructors of the Inductive data type
are the possible inhabitants of the type. When pattern matching onto an
inductive data type, information is gained about the type based on the index
corresponding to the constructors produced by the pattern match.

Many constructions within Agda are common enough that they are desirable at all
levels of universe. Agda does not have cumulativity so, to assist universe
generic programming, universe polymorphism can be used.

\todo{unit and void}

A more useful example to consider is the sized list type, or vector

\ExecuteMetaData[../agda/latex/ThesisMisc.tex]{misc-inductive}

The curly braces in the type definition indicate implicit arguments, arguments
Agda will try and infer from other arguments. \AgdaDatatype{Vec} features set
polymorphism, parameters and indices. The two constructors for
\AgdaDatatype{Vec}, are indexed by their size. The empty list has size
\AgdaInductiveConstructor{zero} to indicate its emptiness.
\AgdaOperator{\AgdaInductiveConstructor{\AgdaUnderscore{}∷\AgdaUnderscore{}}}
takes an element of type \verb|a| and a Vector of size \verb|n| and appends the
singleton to the beginning creating a vector of size
\AgdaInductiveConstructor{suc} \verb|n|. When pattern matching onto a
\AgdaDatatype{Vec}, information is gained about the inductive argument to the
type i.e for the empty list has size \AgdaInductiveConstructor{zero} and in the
inductive case that the list has size \AgdaInductiveConstructor{suc} \verb|n|
for some \verb|n|. The remaining important types of \mlt type theory can
now be introduced as inductive types beginning with Sigma types.
\todo{mixfix syntax}

\ExecuteMetaData[../agda/latex/ThesisMisc.tex]{misc-sigma}

The output universe for sigma must be the maximum of the levels of the first and
second components of the sigma type which can be done using \todo{lub}. The
product type is defined similarly.

\ExecuteMetaData[../agda/latex/ThesisMisc.tex]{misc-product}

and coproducts

\ExecuteMetaData[../agda/latex/ThesisMisc.tex]{misc-coproduct}

Instead of introducing induction and recursion principles for inductive data
types, Agda instead opts for deep pattern matching whereby an inductive datatype
can be expanded into its constitutent components. This is justified by the fact
that all inductive data types possible of being defined within (normal) Agda
correspond to W types i.e. types that admit a well-founded induction principle.
An example of pattern matching can be used to define a projection out of product
types.

\ExecuteMetaData[../agda/latex/ThesisMisc.tex]{misc-projl}

\todo{pattern match that introduces information}

The final type needed before propositions-as-types can be employed is the
equality type which is defined as follows

\ExecuteMetaData[../agda/latex/ThesisMisc.tex]{misc-equality}

For each inhabitant of \verb|A|, \verb|x|, there is unique inhabitant of the
equality type parameterised by \verb|x| which is when the second indexed
argument to the equality type normalises to the first parameter. This
restriction can only be made if the second argument is an index so we are able
to restrict its shape.

Proving theorems with no additional features in the language would prove
difficult. Mathematical structures would be a pain to define in the standard
inductive style as often they will often consist of a single constructor with
arguments that depend on each other. To assist with these types of structure
record types exist. Records are extensions of $\Sigma$-types which have named
fields to assist with referring to the individual components, an illustrative
example is the definition of a monoid, $(S \, , \, \bullet , \, e)$, within Agda

\ExecuteMetaData[../agda/latex/ThesisMisc.tex]{misc-monoid}

where \AgdaDatatype{lsuc} is the equivalent of \AgdaDatatype{suc} but for
levels. \todo{underscore}This is required due to \AgdaField{$\bullet$} which
forces the implicit constructor for the type to be of a higher sort than
\AgdaField{S} itself, due to the predicativity of the underlying type theory.
The identifiers to the left hand side of the colon under the \AgdaKeyword{field}
keyword in the above definitions define projections out of a
\AgdaDatatype{Monoid} object. The first three fields correspond to the elements
of a tuple representing a monoid, the underlying set, binary operation and
identity element.

Another limitation within the currently outlined framework with respect to
proving theorem is the definition of the equality type. In a world where where
Agda used induction principles instead of pattern matching, \textsf{Axiom J},
would, in some sense, not be strong enough to be useful when proving a
significant number of theorems. The problem, being addressed can be introduced
by considering a homomorphism type for the above definition of monoid

\begin{AgdaMultiCode}
\ExecuteMetaData[../agda/latex/ThesisMisc.tex]{misc-monhom}
\ExecuteMetaData[../agda/latex/ThesisMisc.tex]{misc-monhom1}
\end{AgdaMultiCode}

where the fields of the second monoid are postfixed with an apostrophe. Consider
showing two monoids are propositionally equal. For records, this amounts to
showing that its fields are equal i.e that the underlying functions are the
same but \textit{also} the proofs of preservation of identity and and operation
are the same. The proof-relevant nature of the underlying type theory enables
two different proofs to be distinguished by their normal form. It would be
unreasonable and beside-the-point to demand this when equating two monoids but
it is \textit{required} when equating using propositional equality. This is a
well understood problem with various solutions \todo{find solutions}. The
approach taken in standard Agda is to employ an additional axiom on the identity
type known as Streicher's \textsf{Axiom K} \cite{streicher1993investigations}.
\textsf{Axiom K} is an axiom that enables it to be proven that all inhabitants
of the identity type are \textsf{refl}. Introducing \textsf{Axiom K} into MLTT
turns the theory into a proof-irrelevant one. The proofs for both structures
can be reduced to refl and then equated propositionally and all that remains is
showing equality of functions. Introducing \textsf{Axiom K} is not without its
downsides however, recent advancements in type theory \cite{hottbook}
show that there are signifcant advantages to working within a proof-relevant
setting which are inconsistent with \textsf{Axiom K} which are discussed
\todo{put where discussed and write axiom K}.

The last limitation that must be addressed is pertinent to the goal of
formalising category theory within type theory. Returning again to considering
equality of monoid homomorphisms, by employing \textsf{Axiom K}, showing
equality of monoid homomorphisms amounts solely to showing equality of
functions. Within set theory, it is common to equate functions that are
pointwise equal

\begin{align*}
    (\forall x \; f(x) = g(x)) \implies f=g
\end{align*}

This notion of equality ignores, for better or worse, the computational content
of the individual functions. It does not matter if functions are operationally
different but only that they are functionally different. This principle is not
derivable within standard MLTT and therefore, if it is going to be used, it must
be postulated as an axiom. For many, a distinct advantage of computer-aided
theorem proving using types is its intrinsically constructive interpretation and
therefore it is common to avoid axioms such as function extensionality within
the type theory.


A common solution within the type theory of unmodified Agda is the setoid
approach. A setoid is a set or type alongside an equivalence relation \todo{add
setoid def}. Setoids can be used to work with and prove properties of
extensional equalities without introducing a new axiom. This is pertinent to
category theory as, often, structures like monoids and monoid homorphisms are
the subject of examination and a useful notion of equality is essential for
making progress. Proofs with respect to the equivalence relation can be done in
a largely similar way to using propositional equality using the properties of
reflexivity, transitivity and symmetry of the equivalence relation \todo{monoid
example}.  Other situations in which it is preferable to use setoids is when
working with algebraic structures where the underlying equality is not truly
propositional equality for example the monoid consisting of the rational numbers
under addition. Commonly the rational numbers are defined as a pair of integers
however equality on fractions usually consists of equality of their reduced
forms, generating an equivalence class of fractions. In general quotiented
structures are not readily available within standard MLTT.

The primary limitation when opting for setoid equality over propositionally
equality is the inability to employ indecernability of identicals or congruence
a natural consequence of \textsf{Axiom J}. This is an advantage of
propositionally equality since this property must be proven for each equivalence
relation separately, in some circumstances adding on a significant amount of
work. Recent advancements in type theory have produced a type theory in which
function extensionality can be derived and has computational content, discussed
in \todo{hott section}.

\section{Category Theory}

\subsection{Basic Definitions}
Category theory is a unifying field of mathematics that examines abstract
structure. A category, \textbf{C}, is a mathematical structure containing a
class objects and a class of morphisms or directed relations between said
objects. Many expositions of category theory are somewhat vague surrounding
exactly mathematical structures that objects and arrows are. If objects and
arrows are \textit{sets} of things, swathes of mathematical objects are
inaccessible to category theory because they are too large due to Russell's
paradox style problems such as a category of all sets or a category of all
categories. This was a main motivation behind \mlt's theory of types by
defining categories and the objects and arrows of categories as types. By paying
closer attention to what is in the metatheory versus internal to theory i.e. set
membership versus a typing judgement and by paying closer attention to the
predicativity of the system in question, \textit{larger} objects can safely be
embedded in the system\todo{explain more}. In the informal presentation of
category theory, as is prevelant in the literature, no firm foundations will be
provided so as to aid in understanding. This will be followed by a formal
presentation of the constructions within Agda. To repeat, a category is a
collection of objects and a collection of arrows between objects.
\begin{align*}
    &\textrm{Obj} : A \\
    &\textrm{Arr} : A \rightarrow B
\end{align*}

In Agda, categories can be constructed in with relative simplicity. The initial
concepts introduced here are taken from \verb|cats|, a Category Theory library
in Agda by Jannis Limpberg. \AgdaBound{lo}, \AgdaBound{la} and \AgdaBound{l≈}
are used in the following section as variables of type \AgdaDatatype{Level}.

Categories can be introduced as a record parameterised by the level of their
objects, arrows and type of morphism equality respectively

\ExecuteMetaData[../agda/latex/ThesisCategory.tex]{cat-def}

Categories are typed at a level above the largest of objects, arrows and
equalities in order to present equality on morphisms as relations on types as
per the Agda standard library. In addition to objects and arrows, there exists a
binary operation on morphisms known as composition which takes two morphisms,
$f : A \rightarrow B$ and $g : B \rightarrow C$ and produces a third morphism
$g \circ f : A \rightarrow C$. Furthemore, for each object, $A$, in the
category, there exists an identity arrow $id_{A} : A \rightarrow A$.
\begin{align*}
    &\textrm{Identity} : \forall A \in \textrm{Obj}(\textbf{C}) \, \, \exists \, \, id : A \rightarrow A \\
    &\textrm{Composition} : \textrm{Given } f : A \rightarrow B
    \textrm{ and } g : B \rightarrow C \, \exists \, \, g \circ f : A \rightarrow C
\end{align*}

In Agda, identity arrows and composition take their obvious definitions. The identity
arrow provides a distinguished morphism for each object implicitly and
composition is a function that takes two morphisms of the correct shape and
returns the appropriate morphism
\ExecuteMetaData[../agda/latex/ThesisCategory.tex]{cat-field-comp-id}

The current operations defined for a category must adhere to a few more axioms,
namely
\begin{align*}
    &\textrm{Neutrality of identity} : \forall \, g : A \rightarrow B \, \, \, \,
    g \circ id_{A} = g \textrm{  and    } \forall f : C \rightarrow D \, \,  \, \,
    id_{D} \circ f = f \\
    &\textrm{Associativity of Composition} : \forall \, f \, , g \, \, , \,  h \, \, \,
    \, \, ((h \circ g) \circ f) = (h \circ (g \circ f))
\end{align*}

As mentioned in the previous section, when codifying equalities on morphisms,
such as associativity of composition and the neutrality of identity within
Agda, it is often not practical to use propositional equality. It is common to
work with categories with which the morphisms are functions between types
equipped with additional structure. To work with these in standard Agda, either
extensionality must be postulated or setoids must be used. There are other
factors involved with the decision between equality on morphsims being
propositional or setoid such as performance and ease-of-use, some of which is
discussed in.

\ExecuteMetaData[../agda/latex/ThesisCategory.tex]{cat-field-rel}

The \AgdaFunction{IsEquivalence} function establishes the appropriate proofs of
reflexivity, transitivity and symmetry.

With the notion of equality of morphisms in place it is possible to state the
properties of composition and identity

\ExecuteMetaData[../agda/latex/ThesisCategory.tex]{cat-field-comp}

The field \AgdaField{∘{-}resp} is the result of the aformentioned lack of
fongruence for equivalence relations. Preserves2 indicates that composition is
congruent in both of its arguments. This allows us to target individual
compositions in a large categorical term to apply an equality. This is given for
free when using propositional equality as functions are unable to distinguish
terms with the same normal form. This can be seen as one of the downsides to
using equivalence relations as congruence must be proven for every each
individual equivalence relation.\todo{congruence}

There are many examples of categories throughout mathematics and computing. The
category of groups, \textbf{Group}, has as its objects groups and its morphisms
group homomorphisms. The category of sets, \textbf{Set}, has as its objects sets
and its morphsisms total functions. A category of types will be introduced in
\ldots which will be used to explore an application of lawvere's theorem
\subsection{Universal Constructions}

A key idea of category theory are universal constructions. Universal
constructions are common patterns that occur throughout mathematics
that aim to capture the essence of these patterns at the categorical level. The
universal constructions presented here are those that will be of use within the
thesis.

\subsubsection{Terminal Objects}
Terminal objects are constructions that capture the minimal structure required
to be an object within a category. They often correspond to the trivial examples
of objects within the category.  A terminal object of a category \textbf{C} is
an object, $T$, such that, for all other objects, $A$ in the category, there
exists a unique arrow $!_{A}: A \rightarrow T$. This can be shown as a diagram
where the dashed line indicates uniqueness.

\[\begin{tikzcd}[sep=huge]
A \arrow[d, "!_A", dashed] \\
T
\end{tikzcd}\]

As is common with universal constructions, terminal objects in categories are
unique up to unique isomorphism. Examples of terminal objects in common
categories include any singleton in \textbf{Set} and the one element group in
\textbf{Group} Formalising universal constructions within Agda requires the
notion of unique arrow

\ExecuteMetaData[../agda/latex/ThesisUnique.tex]{unique-def-unique}

Uniqueness is often given with respect to a property (hence universal
properties). In Agda this amounts to formulating the property as type
parameterised by the property and an arrow satisfying the property. The type
encodes a function which, given any other arrow satisfying the property
expresses equality to the parameterised arrow. Using this a general unique arrow
can be encoded using a trivial function that always returns the unit. Universal
properties can now be given as an object, a proposition and a proof of
uniqueness

\ExecuteMetaData[../agda/latex/ThesisUnique.tex]{unique-def-exun}

Agda has support for custom syntax directives which can be used to create a
universal mapping type postulating the existence of a unique arrow. Below is an
example of defining products using this where \todo{grab exists}, desugars to
the universal property type above.

\ExecuteMetaData[../agda/latex/ThesisUnique.tex]{unique-def-terminal}

A category having a terminal object can now be encoded as a proposition which
takes a category and provides an object alongside a proof that it is terminal

\ExecuteMetaData[../agda/latex/ThesisTerminal.tex]{terminal-def-has}

Lawvere's fixed point theorem is a theorem about cartesian closed categories.
Cartesian closed categories are categories with three specific universal
properties, a terminal object, binary products and exponentials. Having a terminal
object and binary products is also called having finite products. Therefore the
definition of a CCC in Agda is appropriately.

\ExecuteMetaData[../agda/latex/ThesisCCC.tex]{ccc-def-is-ccc}

\subsubsection{Products}
Products exemplify a common construction in categories of combining the
structure of two objects (in some canonical way) within the category to produce
an object of the same category. In more concrete terms the product of two
objects $A$ and $B$ in the category $\textbf{C}$ is an triple $(A \times B,
\pi_{1}, \pi_{2})$ where for all other objects $C$ in $\textbf{C}$ with
projections $f: C \rightarrow A$ and $g: C \rightarrow B$ the unique arrow
$\langle f, g\rangle : C \rightarrow A \times B$ can be formed such that the
following diagram commutes:

\[\begin{tikzcd}[sep=huge]
 & C \arrow[ld, "f"'] \arrow[rd, "g"] \arrow[d, "{\langle f, g\rangle}" description, dashed] &  \\
A & A \times B \arrow[l, "\pi_2"] \arrow[r, "\pi_1"'] & B
\end{tikzcd}\]

where the dashed arrow indicates uniqueness. This can be extended to
\textit{n}-ary products in the obvious way.

As with terminal objects, products are unique up to unique isomorphism. Examples
of products within familiar categories include the cartesian product $\times$ in
\textbf{Set}, defined as the set of all tuples of elements from two separate
sets. If products can be formed for every finite set of objects in a category it
is said to be cartesian.

Products have a slightly more involved definition than terminal objects. Beginning
with the uniqueness principle for products

\ExecuteMetaData[../agda/latex/ThesisProduct.tex]{product-def-product}

\AgdaFunction{IsProduct} takes an indexing function, a product object, and some
projections out of the product into the components of the indexing category.
\AgdaFunction{IsProduct} returns a function type which, upon being supplied a
set of projections from an object to the indexing set, returns a unique arrow
from the the object to the previously supplied product object satisfying the
commuting diagrams for the product. A product object for a given indexed family
of objects \verb|O| can be defined as the product object itself,
\AgdaField{prod}, the projections out of the product object, \AgdaField{proj},
and the proof of uniqueness, \AgdaField{isProduct}.

\ExecuteMetaData[../agda/latex/ThesisProduct.tex]{product-def-prod}

A binary product is a product where the function that indexes the family of
objects is the a boolean elimination function

\ExecuteMetaData[../agda/latex/ThesisProduct.tex]{product-def-bool}
\ExecuteMetaData[../agda/latex/ThesisProduct.tex]{product-def-binprod}

A category containing binary products can be encoded as a category equipped with
a operation that, for every pair of objects, \verb|A| and \verb|B|, produces the
product object for the pair.

\begin{AgdaMultiCode}
\ExecuteMetaData[../agda/latex/ThesisBinProd.tex]{binprod-has-binary-products}
\ExecuteMetaData[../agda/latex/ThesisBinProd.tex]{binprod-times}
\\
For notational convenience a function that returns the object within the
category from a particular product object is useful.\\
\ExecuteMetaData[../agda/latex/ThesisBinProd.tex]{binprod-obj}
\\
Also useful are the projections out of the product \\
\ExecuteMetaData[../agda/latex/ThesisBinProd.tex]{binprod-projr}
\ExecuteMetaData[../agda/latex/ThesisBinProd.tex]{binprod-projl} \\
and the methods of forming the unique arrows \\
\ExecuteMetaData[../agda/latex/ThesisBinProd.tex]{binprod-unique-arr}
\ExecuteMetaData[../agda/latex/ThesisBinProd.tex]{binprod-unique-pair}
\end{AgdaMultiCode}

Examples of products include the direct-product of groups in \textbf{Group} and
the cartesian product of sets in \textbf{Set}. A category with a terminal object
and products is said to have finite products and is a cartesian category.

\subsubsection{Exponentials}

Exponential objects are universal constructions that capture the notion of
function spaces or higher order objects. The exponential, $B^{A}$, indicates the
mappings from the object $A$ to $B$. This is paired with the morphism $eval:
B^{A} \times A \rightarrow B$ such that for any object $Z$ and morphism $f :
Z\times A \rightarrow B$ there exists a unique morphism $\tilde{f}: Z
\rightarrow B^{A}$ such that the following diagram commutes:

\[\begin{tikzcd}[sep=huge]
    B^{A} \times A \arrow[r, "eval"]
    & C \\
    A \times B \arrow[ru, "f"'] \arrow[u, "\langle \: \hat{f} \: \times \: id
    \rangle"] &
\end{tikzcd}\]

$\tilde{f}$, or transposition, can be thought of as currying in the functional
programming sense, taking a function in multiple arguments to a sequence of
functions in one argument.

In Agda, an exponential object for objects \verb|B| and \verb|C| consist of an
object

\begin{AgdaMultiCode}
\ExecuteMetaData[../agda/latex/ThesisExponential.tex]{expon-exp}
\\
the evaluation map
\\
\ExecuteMetaData[../agda/latex/ThesisExponential.tex]{expon-eval}
\\
and the uniquness principle for products
\\
\ExecuteMetaData[../agda/latex/ThesisExponential.tex]{expon-unique}\\
For ease of use the function producing the unique arrow can be extracted.\\
\ExecuteMetaData[../agda/latex/ThesisExponential.tex]{expon-curry}
\end{AgdaMultiCode}

A category that has exponentials is one that  where the exponetial object can be
formed for every pair of objects

\begin{AgdaMultiCode}
\ExecuteMetaData[../agda/latex/ThesisHasExp.tex]{hasexp-def}
\ExecuteMetaData[../agda/latex/ThesisHasExp.tex]{hasexp-exp}
\\
A convenient exponential operation can be defined that extracts the object
\\
\ExecuteMetaData[../agda/latex/ThesisHasExp.tex]{hasexp-expo}
\\
With a generic evaluation map
\\
\ExecuteMetaData[../agda/latex/ThesisHasExp.tex]{hasexp-eval}
\\
and \AgdaFunction{curry} and \AgdaFunction{uncurry} as isomorphisms
\\
\ExecuteMetaData[../agda/latex/ThesisHasExp.tex]{hasexp-curry}
\ExecuteMetaData[../agda/latex/ThesisHasExp.tex]{hasexp-uncurry}
\\
\end{AgdaMultiCode}


\todo{law fix}


% -----------------------------------------------------------------------------

\chapter{Project Execution}
\label{chap:execution}

% !TEX root = ./dissertation.tex
In this section the main contributions of this thesis will be presented.
\section{Points}

\subsection{Points} \todo{expand proofs further}
All that follows is work done during this thesis. The theorem makes use of the
notion of points and some surrounding definitions

Points are a categorical abstraction that generalise the notion of elements of
a set. A point is an arrow from the terminal object to any other object

\ExecuteMetaData[../agda/latex/ThesisPoints.tex]{point-def-point}

Given the name of Lawvere's theorem it makes sense to formalise the notion of a
fixed point categorically

\ExecuteMetaData[../agda/latex/ThesisPoints.tex]{point-def-fixed-point}

A fixed point of a morphism is a point that is idempotent under composition to
the right with the morphism. As is common within the \verb|cats| library this is
then wrapped up in a record as a sigma type to express that a given function has
a fixed point

\ExecuteMetaData[../agda/latex/ThesisPoints.tex]{point-def-has-fixed-point}

Now the fixed point property which features in Lawvere's theorem can be
formalised which is a predicate on a object in a category expressing that all
endomorphisms on the object have a fixed point.

\ExecuteMetaData[../agda/latex/ThesisPoints.tex]{point-def-fixed-point-property}
 on the object have a fixed point.

Another useful feature in lawvere's theorem is the notion of point surjectivity
which itself requires some machinery. First the notion of a solved equation with
points

\ExecuteMetaData[../agda/latex/ThesisPoints.tex]{point-def-solution}

Point surjectivity expresses the notion that given a point to $B$, $b : 1
\rightarrow B$, and a morphism $f : A \rightarrow B$, we can produce a point to
$A$, $a : 1 \rightarrow A$, that satisfies the equation $f \circ a = b$.

Packaging this up into a sigma type which, given a morphism from an object A to
an object B and a point to B contains a point to A and a proof that is
constitutes a solution to the triple.

\ExecuteMetaData[../agda/latex/ThesisPoints.tex]{point-def-has-solution}

A point surjective morphism is a function for which every point to B there
exists a solution

\ExecuteMetaData[../agda/latex/ThesisPoints.tex]{point-def-is-point-surjective}

The formulation of the point surjectivity used in the theorem as a record
confirming the existence of a point surjective function between two objects

\ExecuteMetaData[../agda/latex/ThesisPoints.tex]{point-def-point-surjective}

\section{Lawvere's Fixed Point Theorem}

It is now possible to state Lawvere's theorem precisely, working within a
Cartesian Closed Category.

\ExecuteMetaData[../agda/latex/ThesisDiagonal.tex]{diagonal-type-diagonal}

Or mathematically that in a Cartesian Closed Category, given a point-surjective
function from some object $A$ to the exponential object, $B^A$, from $A$ to some object
$B$, every endomorphism on $B$ has a fixed point.

The proof of the theorem will be developed line by line. The first step is to
pattern match on the arguments to the proof and bring in the constructors for
the output type.

\begin{AgdaMultiCode}
\ExecuteMetaData[../agda/latex/ThesisDiagonal.tex]{diagonal-pattern}
\ExecuteMetaData[../agda/latex/ThesisDiagonal.tex]{diagonal-pattern-end}
\end{AgdaMultiCode}

The first argument to the proof is the point-surjective morphism constituting
the morphism and the proof of point-surjectivity, and the second argument is the
endomorphism on \verb|B|. The output, a record with holes, requires a point to
\verb|B| alongside a proof that it is a fixed point of \verb|f|.

To produce a fixed-point the goal is to create a morphism, \verb|g|,  from \verb|A| to
\verb|B| and then exploit the point-surjective morphism to find a point to
\verb|A|. With the correctly chosen \verb|g| the composition of this point to A
with \verb|g| will be a fixed-point. \verb|g| will be constructed such that is
in some way "self-replicating".

\ExecuteMetaData[../agda/latex/ThesisDiagonal.tex]{diagonal-h-def}

Categorically, \verb|g| represents the following diagram

\[\begin{tikzcd}[sep=huge]
    B & B \arrow[l, "f"] & B^A \times A \arrow[l, "eval"] & A \times A \arrow[l,
    "{\langle \phi \, , id\rangle}"] & A \arrow[l, "\delta"]
\end{tikzcd}\]

In order to push \verb|g| morphism back through the point-surjective morphism
it needs to be turned into a point to the exponential object. This can be
achieved via two isomorphisms, $1 \times A \cong A$ and $\textnormal{hom}(A \times B \: , \, C)
\cong \textnormal{hom}(A \: , \, C^B)$, the types and directions that are used proof as
follows

\begin{AgdaMultiCode}
\ExecuteMetaData[../agda/latex/ThesisExtension.tex]{extension-types-collapseToOne}
\ExecuteMetaData[../agda/latex/ThesisTypes.tex]{types-type-curry}
\end{AgdaMultiCode}

By applying the first isomorphism followed by the second, a point to $B^A$ can
be acquired
\begin{AgdaAlign}
\ExecuteMetaData[../agda/latex/ThesisDiagonal.tex]{diagonal-g'-def}

The point-surjectivity of $\phi$ can now be used to acquire the associated point
to \verb|A| with \verb|g'|

\ExecuteMetaData[../agda/latex/ThesisDiagonal.tex]{diagonal-ps-def}
\end{AgdaAlign}

The fixed point construction can now be achieved by composing $\phi$ with
\verb|u| twice to obtain a point to \verb|B|. After composing once with \verb|u|
a point to $B^A$ is obtained. This must be pushed through the aformentioned
isomorphisms to get a morphism, $A \rightarrow B$, to compose with \verb|u|
again.

\ExecuteMetaData[../agda/latex/ThesisDiagonal.tex]{diagonal-isos-proof}

Now it must be shown that \verb|f| $\circ$ \verb|fixedPoint| $\approx$
\verb|fixedPoint|. This proof will be developed using equational reasoning. The
proof starts with the word \AgdaFunction{begin} and the left-hand side of the
equality, with expressions separated by
equalities on morphisms, put inside
\AgdaOperator{\AgdaFunction{≈⟨}}\AgdaSpace{}\AgdaOperator{\AgdaFunction{⟩}}, and
ends with the right-hand side of the equality followed by
\AgdaOperator{\AgdaFunction{∎}}. In the case of the required proof

\ExecuteMetaData[../agda/latex/ThesisDiagonal.tex]{diagonal-fix-proof}

The first transformation in the proof is to use the point-surjectivity of
$\phi$ to expand the the $\phi$ $\circ$ \verb|u| within the definition of
\verb|fixedPoint| to \verb|g'|.

The proof of point-surjectivity is extracted as follows

\ExecuteMetaData[../agda/latex/ThesisDiagonal.tex]{diagonal-ps-proof}

This cannot be used directly due to the application of \AgdaFunction{curry} and
\AgdaFunction{extendToOne} to the expression. The usage of equivlance relations
means that congruence must be proved separately for every function on
morphisms. These two proofs have the following types but the proofs are elided
due to unecessary complexity.

\begin{AgdaSuppressSpace}
\ExecuteMetaData[../agda/latex/ThesisDiagonal.tex]{diagonal-uncurry-resp-prf}
\ExecuteMetaData[../agda/latex/ThesisExtension.tex]{extension-coll21-r-type}
\end{AgdaSuppressSpace}

To make \verb|ps-proof| work within the nested function applications they are
wrapped in the two proofs of congruence necessary

\ExecuteMetaData[../agda/latex/ThesisDiagonal.tex]{diagonal-col-unc-ps}

This can then be used by targeting the lefthand morphism of the outermost
composition and change this to \verb|g'|. This is done using
\AgdaFunction{∘-resp-l} which allows a proof to be applied to chang the lefthand
side of a morphism composition.

\begin{AgdaMultiCode}
\ExecuteMetaData[../agda/latex/ThesisDiagonal.tex]{diagonal-col-unc-ps-trans}
\ExecuteMetaData[../agda/latex/ThesisDiagonal.tex]{diagonal-col-unc-ps-trans2}
\\
The next transformation is accomplished by utilising that \AgdaFunction{curry}
is an isomorphism with respect to \AgdaFunction{uncurry}, and that
\AgdaFunction{collapseToOne} is an isomorphism with respect to
\AgdaFunction{extendToOne}. With this \verb|g'| $\circ$ \verb|u| is obtained,
expanding \verb|g'|, \\
\ExecuteMetaData[../agda/latex/ThesisDiagonal.tex]{diagonal-col-unc-ps-trans2}
\ExecuteMetaData[../agda/latex/ThesisDiagonal.tex]{diagonal-g'-reduc}
\ExecuteMetaData[../agda/latex/ThesisDiagonal.tex]{diagonal-g'-reduc1}
\\
Before begin able to manipulate this expression the expression must be
reassociated. This is particularly tedious.\\

\ExecuteMetaData[../agda/latex/ThesisDiagonal.tex]{diagonal-g'-reduc1}
\ExecuteMetaData[../agda/latex/ThesisDiagonal.tex]{diagonal-reassoc}
\ExecuteMetaData[../agda/latex/ThesisDiagonal.tex]{diagonal-reassoc1}
\\
Once this has been achieved, definitions can be expanded and applied.
\AgdaFunction{$\delta$} is the unit of the diagonal-product adjunction and can
defined simply as \\
\ExecuteMetaData[../agda/latex/ThesisDiagonal.tex]{diagonal-delta} \\
Intuitively, it can be seen that the precomposition of a morphism by an arrow to
a product object can fused to push the precomposed morphism into each branch of
the product object.
\\
\ExecuteMetaData[../agda/latex/ThesisDiagonal.tex]{diagonal-reassoc1}
\ExecuteMetaData[../agda/latex/ThesisDiagonal.tex]{diagonal-in-u}
\ExecuteMetaData[../agda/latex/ThesisDiagonal.tex]{diagonal-in-u1} \\
To reproduce the original fixed point, $\phi$ should be composed with \verb|u|
followed by another composition with \verb|u|. To do this, a corollary to the
universal property of exponentials must be used, however some rearrangement must
be done. The corollary that needs to be used is \todo{eval-curry}

This requires considerably more work as each transformation of product
objects must be made explicit \\
\ExecuteMetaData[../agda/latex/ThesisDiagonal.tex]{diagonal-in-u1}
\ExecuteMetaData[../agda/latex/ThesisDiagonal.tex]{diagonal-product-rearr}
\ExecuteMetaData[../agda/latex/ThesisDiagonal.tex]{diagonal-product-rearr1} \\
One \verb|u| must be brought into the left-hand product without the right-hand
one in order to match the universal property of exponentials. Another
requirement for \AgdaFunction{eval-curry} is that $\phi$ $\circ$ \verb|u| must
be wrapped inside \AgdaFunction{curry}. This can be done by applying the
\AgdaFunction{curry∘uncurry} isomorphism \\
\ExecuteMetaData[../agda/latex/ThesisDiagonal.tex]{diagonal-product-rearr1}
\ExecuteMetaData[../agda/latex/ThesisDiagonal.tex]{diagonal-curryuncurry}
\ExecuteMetaData[../agda/latex/ThesisDiagonal.tex]{diagonal-curryuncurry1} \\
Now, the universal property can be applied to extract \AgdaFunction{uncurry}
($\phi$ $\circ$ \verb|u|) from the product object \\
\ExecuteMetaData[../agda/latex/ThesisDiagonal.tex]{diagonal-curryuncurry1}
\ExecuteMetaData[../agda/latex/ThesisDiagonal.tex]{diagonal-eval-curry}
\ExecuteMetaData[../agda/latex/ThesisDiagonal.tex]{diagonal-eval-curry1} \\
The end is in sight and all that remains is to extract the second \verb|u|. This
collapsing $A \times 1$ to $A$ which can be achieved by inserting the identity
for $A \times 1$ and deconstructing this into the morphisms comprising the
isomorphism and seeing what happens \\
\ExecuteMetaData[../agda/latex/ThesisDiagonal.tex]{diagonal-eval-curry1}
\ExecuteMetaData[../agda/latex/ThesisDiagonal.tex]{diagonal-ax1-iso}
\ExecuteMetaData[../agda/latex/ThesisDiagonal.tex]{diagonal-ax1-iso1} \\
The isomorphisms happen to precisecly be what is needed to recover the fixed
point \\
\ExecuteMetaData[../agda/latex/ThesisDiagonal.tex]{diagonal-ax1-iso1}
\ExecuteMetaData[../agda/latex/ThesisDiagonal.tex]{diagonal-iso-app}
\ExecuteMetaData[../agda/latex/ThesisDiagonal.tex]{diagonal-iso-app1} \\
Applying \AgdaFunction{projr} to the product object  extracts \verb|u| giving us
the fixed point \\
\ExecuteMetaData[../agda/latex/ThesisDiagonal.tex]{diagonal-proj-out}
\ExecuteMetaData[../agda/latex/ThesisDiagonal.tex]{diagonal-end} \\
The proof can be finished off by filling in the holes in the record \\
\ExecuteMetaData[../agda/latex/ThesisDiagonal.tex]{diagonal-record}
\end{AgdaMultiCode}

The contrapositive of the statement is worth defining as it is useful for some
of the applications.
\ExecuteMetaData[../agda/latex/ThesisDiagonal.tex]{diagonal-cantor}

\section{Applications}

Lawvere's fixed point theorem is an incredibly broad ranging theorem that
generalises many important theorems in mathematical logic and foundational
computer science and mathematics. This thesis will formalise and axiomatise two
specific instances, Cantor's diagonal argument and the first recursion theorem
in the $\lambda$-calculus. An analog to Cantor's diagonal argument will be introduced
alongside a category of small types in place of the category of sets.
After, categorical models of the $\lambda$-calculus will be explored and the
consequences of lawvere's fixed point theorem in these models. This theorem has
many more applications in paradoxes and logic which can be found in (Yanofsky).
Some limitations in formalising these within theorem provers will be explored in
the further work section.
\section{Cantor's Theorem}
An analogue of Cantor's theorem can be constructed using a category of small
types (i.e. all types that belong to a given universe).


A category can be constructed from the elements of any Agda universe of a
particular level.

\ExecuteMetaData[../agda/latex/ThesisSets.tex]{sets-instance}

The objects of the category are the types at the specified level. The morphisms
are the functions between types of this level. Equality of morphisms is
extensional propositional equality for functions. Morphism composition is
function composition, identity morphsism are identity functions. The axioms of
categories come simply from the definition of propositional equality.

\ExecuteMetaData[../agda/latex/ThesisSets.tex]{sets-equality}

Each instance of \AgdaDatatype{Sets} forms a cartesian closed categories.
Working with the lowest level universe.

\ExecuteMetaData[../agda/latex/ThesisCantor.tex]{cantor-univ}

Products correspond to the pair type

\ExecuteMetaData[../agda/latex/ThesisCantor.tex]{cantor-pair}

With the indexed family of projections being the projections out of the pair.

\ExecuteMetaData[../agda/latex/ThesisCantor.tex]{cantor-proj}

All that remains is to prove that these constitute the product object i.e. that
for each pair of arrows from a type to each component of the pair there exists a
(extensionally propositionally) unique function from the type to the pair object.

\ExecuteMetaData[../agda/latex/ThesisCantor.tex]{cantor-unique-type}

Given the indexed-family the unique arrow \verb|u| can be produced by extracting
the values from the indexed family and creating a pair from them

\ExecuteMetaData[../agda/latex/ThesisCantor.tex]{cantor-unique-def}

Where \todo{consturcot} is the constructor for a univeral mapping.

The next field of the universal mapping type is the proof that the provided
arrow satisfies the definition required from the product i.e.

\ExecuteMetaData[../agda/latex/ThesisCantor.tex]{cantor-proj-sat}

or in straightforward mathematical terms that the constructed arrow, \verb|u|,
satisfies with abuse of notation,
\begin{align*}
    \pi_{1} \circ \texttt{u} = f \land \pi_{2} \circ \texttt{u} = g
\end{align*}

where $f$ and $g$ are the two morphisms underlying \verb|p|. This proves to be
trivially true given the definition of \AgdaFunction{proj-pair}.

\ExecuteMetaData[../agda/latex/ThesisCantor.tex]{cantor-unique-def1}

The final field of the univeral mapping constructor is the proof that \verb|u|
is unique i .e. that for any \verb|g|\todo{type} such that \verb|g| satisfies
\AgdaFunction{proj-sat-univ} it is (extensionally) equal to \verb|u| or as a
type

\ExecuteMetaData[../agda/latex/ThesisCantor.tex]{cantor-unique}

The condition \verb|h|, equates \verb|g| to \todo{projpair}. This condition is
something that the unique arrow satisfies by definition. Using the transitivity
of equality 
productsproductsproducst

For a category to have finite products it must also have a terminal object. The
terminal object in the category \AgdaDatatype{Sets1} is the unit type

\ExecuteMetaData[../agda/latex/ThesisUnit.tex]{unit-unit}

To prove that \todo{here} is in fact the terminal object the universal property
must be proven

\ExecuteMetaData[../agda/latex/ThesisCantor.tex]{cantor-terminal-prop1}

This is a universal mapping property. The first argument to the constructor is
the unique arrow and the last the proof of uniqueness. The middle argument
ordinarily corresponds to the property the arrow must satisfy but the property
here is existence and so can be inferred automatically using an underscore as it
is trivially true if the type can be inhabited.

For a given type the function to the terminal object is the function that
constantly returns the single inhabitant of \todo{here}.

\ExecuteMetaData[../agda/latex/ThesisCantor.tex]{cantor-terminal-arrow}

The final component of the terminal object is the proof of uniqueness i.e. that
every function from a type to the terminal object is propositionally
(extensionally) equal. This
is trivially true as there is a single inhabitant of the unit type and therefore
only one place for to which all functions can map.

\ExecuteMetaData[../agda/latex/ThesisCantor.tex]{cantor-terminal-unique}

With this the universal mapping property can be completed

\todo{fix this}
\ExecuteMetaData[../agda/latex/ThesisCantor.tex]{cantor-terminal-prop2}

And it can be established that \AgdaDatatype{Sets1} has a terminal object

\ExecuteMetaData[../agda/latex/ThesisCantor.tex]{cantor-tisterminal}


The last requirement for a cartesian closed category is exponentials. For every
pair of types an exponential object must be produced consisting of a type, an
evaluation map and transposition (currying). The exponential object for two
types \verb|A| and \verb|B| is the function type between the two

\ExecuteMetaData[../agda/latex/ThesisCantor.tex]{cantor-exponential}

The evaluation map takes a pair containing a function from a type \verb|A| to a
type \verb|B| and term of type \verb|A| and returns a \verb|B|. All that is
required here is to unpack the pair and apply the function to the value.

\ExecuteMetaData[../agda/latex/ThesisCantor.tex]{cantor-eval}

The last component that needs to be provided for exponentials is the curry
function which for the category of small types takes the form

\ExecuteMetaData[../agda/latex/ThesisCantor.tex]{cantor-curry'}

Which returns a universal mapping property. The first argument of the universal
mapping property, as is
usual, is the mapping itself which is the curry function.


\ExecuteMetaData[../agda/latex/ThesisCantor.tex]{cantor-curry}

The second argument is a proof that \AgdaFunction{sets-curry} satisfies the
universal property.

\ExecuteMetaData[../agda/latex/ThesisCantor.tex]{cantor-sets-curry'-sat}

Reducing the left hand side of the propositional equality in the above type the
proof can be completed trivially

\ExecuteMetaData[../agda/latex/ThesisCantor.tex]{cantor-sets-curry'-sat1}

Making use of the universal property for products

\ExecuteMetaData[../agda/latex/ThesisCantor.tex]{cantor-pair-prf}

The last component of the unique mapping property is the proof of uniqueness of
the map the type of which is

\ExecuteMetaData[../agda/latex/ThesisCantor.tex]{cantor-curry-uniq-type}

This proof proves to be difficult to complete. \verb|tProof| has type
\todo{type}. This is an extensional proof equating the two desired structures
however, within MLTT, function extensionality is not derivable. Function extensionality
does not lead to inconsistencies within MLTT when postulated and therefore is
done here to allow the proof to proceed.

\ExecuteMetaData[../agda/latex/ThesisCantor.tex]{cantor-postulate}

Where extensionality equates functions that are pointwise equal

\ExecuteMetaData[../agda/latex/ThesisCantor.tex]{cantor-extensionality}

With extensionality the proof can be completed without significant difficulty
throught \verb|tproof|

\ExecuteMetaData[../agda/latex/ThesisCantor.tex]{cantor-curry-uniq1}

The above defined function complete the definition of \AgdaFunction{set-curry'}

\ExecuteMetaData[../agda/latex/ThesisCantor.tex]{cantor-set-curry}

\AgdaDatatype{Sets1} can now be defined to be closed
\ExecuteMetaData[../agda/latex/ThesisCantor.tex]{cantor-hasexp}

With these definitions the CCCness of \AgdaDatatype{Sets1} can be trivially
established.

The category that has been is unfortunately not a valid model of set theory and
is only a toy language for reasons that can be found \ldots \todo{where found}.
A type theoretic analogue of Cantor's theorem can be established showing there
is no point-surjection from a type to the predicates on the type. In set theory,
for a given set, the predicates on the set can be considered as subsets placing
the elements of the powerset of a set in correspondence with the predicates on
the set. With no coherent notion of membership relation or subset relation (by
design) there is no way to faithfully model the Cantor's theorem  within the
category produced. Even so, the proof shows, with enough similarity how the
argument would proceed in the category of sets.

Within \AgdaDatatype{Sets1}, predicates are functions from a type \verb|A| to
\AgdaDatatype{Bool} type.

\ExecuteMetaData[../agda/latex/ThesisBool.tex]{bool-def}

With \AgdaDatatype{Bool}, Cantor's theorem can be reframed within
\AgdaDatatype{Sets1} as

\ExecuteMetaData[../agda/latex/ThesisCantor.tex]{cantor-cantor-type}

Or that, in English, for all types (in \AgdaDatatype{Set}) there does not exist
a point-surjection from the type to the predicates on the type. It is unclear
to the author as to whether point-surjectivity has a more coherent intepretation
within \AgdaDatatype{Sets1}.

To prove this  the contrapositive of
\AgdaFunction{lawvere}, \AgdaFunction{cantor} will be used. As a reminder

\ExecuteMetaData[../agda/latex/ThesisDiagonal.tex]{diagonal-cantor}

To make use of this it is necessary to show that \AgdaDatatype{Bool} does not
have the fixed point property.

\ExecuteMetaData[../agda/latex/ThesisCantor.tex]{cantor-nofixpt}

Recall that the definition of \todo{not} is
\todo{here}

Therefore, given a function which finds the fixed point of any function from
\AgdaDatatype{Bool} to \AgdaDatatype{Bool}, an inhabitant of void must be
provided. The observation that results in Cantor's theorem being an application
of Lawvere's theorem is that there is a function from \AgdaDatatype{Bool} to
\AgdaDatatype{Bool} that does not have a fixed point, the familiar function from
classical logic, negation or \AgdaFunction{not}.

\ExecuteMetaData[../agda/latex/ThesisCantor.tex]{cantor-not}

In fact, with respect to sets, every set with more than one element has at least
one function to itself without a fixed point leading to an extension of Cantor's
theorem i.e. that for all sets $A$ and $B$ with cardinality greater than one
there does not exist a surjective function from $A \rightarrow B^A$. Before
returning to \AgdaFunction{noFixPtBool}, a proof that \AgdaFunction{not} is
needed.

\ExecuteMetaData[../agda/latex/ThesisCantor.tex]{cantor-not-fx-type}

\AgdaFunction{not} \verb|x| reduces to a different normal form to \verb|x| for
both \AgdaInductiveConstructor{true} and \AgdaInductiveConstructor{false} and therefore the absurd pattern can
be introduced in both cases.

\ExecuteMetaData[../agda/latex/ThesisCantor.tex]{cantor-not-fx}
\todo{explain absurd pattern}

\AgdaFunction{not-fx-pt} is now used to derive \AgdaFunction{noFixPtBool}
through a contradiction. \todo{not} \AgdaFunction{noFixPtBool} is a function
which, given a proof that \AgdaDatatype{Bool} has a fixed property, can derive
false. The proof that \AgdaDatatype{Bool} has the fixed point property can be
used to derive a fixed point for \AgdaFunction{not} which we have already proven
does not have a fixed point. This is done through the \AgdaKeyword{with}
construct within Agda. \AgdaKeyword{with} allows an intermediate computation to
be pattern matched on. In the case of \AgdaFunction{noFixPtBool} it is the
result of applying the proof of the fixed proof property \verb|Y|, a fixed point
combinator, to the \AgdaFunction{not} function.

\ExecuteMetaData[../agda/latex/ThesisCantor.tex]{cantor-nofixpt1}

Through this, a fixed-point, \verb|X| of \AgdaFunction{not} can be extracted alongside a
proof of \verb|X| fixed pointedness.

\ExecuteMetaData[../agda/latex/ThesisCantor.tex]{cantor-nof-def}

Now, within the scope of the \AgdaFunction{noFixPtBool}, there exist proofs that
\AgdaFunction{not} both does and doesn't have a fixed point.
\AgdaFunction{not-fx-pt} has type \todo{type}. It is clear that by applying
the proof of the existence of a fixed point to this will derive \todo{bottom} as
needed.

\ExecuteMetaData[../agda/latex/ThesisCantor.tex]{cantor-nof-final}

\AgdaFunction{cantorsDiagonalTheorem} can now be derived as a direct application
of \AgdaFunction{cantor}

\begin{AgdaMultiCode}
    \ExecuteMetaData[../agda/latex/ThesisCantor.tex]{cantor-cantor-type}
    \ExecuteMetaData[../agda/latex/ThesisCantor.tex]{cantor-cantor}
\end{AgdaMultiCode}
\section{The $\lambda$-Calculus}
\todo{lambda equun theory}
There is much to suggest that a coherent interpretation of Lawvere's theorem
exists in the $\lambda$-calculus. The untyped $\lambda$-calculus is famous for
Curry's fixed point combinator \todo{combinator} of which a consequence is that
every $\lambda$-term has a fixed point under application (known as the first
fixed point theorem) indicating that perhaps a direct
application in an appropriate category could yield this result. In addition to
this, significant results in different models of computation are given as a
result as outlined in \todo{contextual background}. This observation has not
gone unnoticed by others. \verb|nLab| \todo{cite}, an online encyclopedia
for category theory, its webpage for Lawvere's fixed point
theorem states in \textbf{Remark 2.6}.

\begin{displayquote}
\textit{Many applications of Lawvere’s fixed point theorem are in the form of negated
propositions, e.g., there is no surjection from a set to its power set, or
Peano arithmetic cannot prove its own consistency. However, there are positive
applications as well, e.g., it implies the existence of fixed-point combinators
in untyped lambda calculus.}
\end{displayquote}

This claim affirms the notion that Curry's fixed combinator could be derived as
an instance of Lawvere's theorem. Despite this remark, the realities of the
situation are not so simple. The \verb|nLab| page does not provide a source for
their remark and, within the literature, there are only informal proofs of this
claim. Upon further examination the informal proofs contain some inaccuracies
that result in them not being a true reflection of Lawvere's theorem in the
context of the untyped $\lambda$-calculus. The precise issues with the
previously presented proofs will be outlined in \todo{evaluation}. In this
thesis a precise account of the relationship indicated within
the remark on \verb|nLab| will be presented, which does not quite extend to establishing the
existence of fixed point combinators within the $\lambda$-calculus but the
result - the first fixed point theorem.

The proof that Lawvere's theorem implies the existence of fixed points for all
$\lambda$-terms rests on the observation first made by Dana Scott in his
development of domain theory, that $\lambda$-terms can be considered as objects
of the models but also as functions between $\lambda$-terms. This naturally
engenders a desire for some object $D$ that is isomorphic to $D^D$, the function
space on $D$. This is impossible for any set within a set theory that rejects
unrestricted comprehension. Dana Scott solved this by constructing an object
$D_{\infty}$ as a complete partial order \todo{here}. These
ideas were generalised by Koymans and Scott identifying that all models of the
$\lambda$-calculus arise from cartesian closed category with an object $D$ that
has a retraction to its own function space, known as a reflexive object.

\todo{name corollary}

The relationship between these constructions and Lawvere's theorem can be
understood by observing that in any $CCC$ with a reflexive object $D$ there is a
point surjective morphism from $D$ to $D^D$, precisely the retraction.
Formalising this in Agda first requires a formalisation of a reflexive object
and retractions. In an arbitrary category a retraction between two objects
\verb|A| and \verb|B| is

\ExecuteMetaData[../agda/latex/ThesisRetract.tex]{retract-def-retract}

i.e. A pair of arrows in both directions between \verb|A| and \verb|B| and a
proof that the composition of the two form the identity in a given direction.
A reflexive object is simply some object \verb|D| alongside a retraction,
\AgdaDatatype{Retract} \verb|D| $\verb|D|^{\verb|D|}$. Now the relevant
corollary to Lawvere's fixed point theorem can be stated precisely, working in a
CCC

\ExecuteMetaData[../agda/latex/ThesisY.tex]{Y-def-corollary}

The proof can be created by creating a point-surjective function from \verb|X|
to $\verb|X|^{\verb|X|}$ and applying the earlier proof of
\AgdaFunction{lawvere}. First, the retraction can be pattern matched on and
Lawvere introduced with the arrow from \verb|X| to $\verb|X|^{\verb|X|}$ with
only a proof of point surjectivity required.

\ExecuteMetaData[../agda/latex/ThesisY.tex]{Y-def-corollary-body}

With \verb|b| being a general point to $\verb|X|^{\verb|X|}$, a point to
\verb|X| needs to be provided alongside a proof that \verb|b| is the solution to
the obvious equation \todo{fill in equation}. The point to \verb|X| is found by
using the retraction. The proof of equality is simply achieved by exploiting the
definition of a retraction to collapse the identity.

\ExecuteMetaData[../agda/latex/ThesisY.tex]{Y-point-surjective}


The implications of this corollary can be understood from this excerpt from section
5 of Barendregt's \textit{The Lambda Calculus: Its Syntax and Semantics}
\cite{barendregt1992lambda} on
models.

\begin{displayquote}
\textit{"\ldots for the construction of a $\lambda$-calculus model it
is sufficient to have an object $D$ in a CCC such that $D^D$ is a retract of
$D$."}
\end{displayquote}

Baredregt's textbook is a useful resource for all work within this section. From
the above quote and the earlier corollary to Lawvere's theorem it can be
concluded that the categorical interpretation of every model of the
$\lambda$-calculus has an object with the fixed point property.

Models, in the model-theoretic sense, are helpful for exploring properties of
the $\lambda$-calculus that are not immediate from the equational theory and
syntax itself. Concrete models of the $\lambda$-calculus include \todo{concrete
models} but what it meant to be model of the $\lambda$-calculus was elusive
until \todo{koymans} where the notion of $\lambda$-algebra was introduced as the
class of structures corresponding to the $\lambda$-calculus.

The definition of $\lambda$-algebras are predicated on the definition of an
applicative structure. An applicative structure is a tuple $M = (A \, , \,
\bullet)$ where $A$ is a set and $\bullet$ is a binary operation on $A$.


A useful type of models to be considered are the class of syntactic models. A
syntactic applicative structure adds a method of intepreting terms in the
$\lambda$-calculus, elements of the set $\Lambda$, into the applicative
struture. In other words a syntactic applicative structure is a triple $M = (A
\, , \, \bullet \, , \, \llbracket \, \rrbracket)$ of an underlying set $A$, a
binary operation $\bullet : A \rightarrow A$ and a syntactical intepretation
function $\llbracket \, \rrbracket$. See Barendregt \todo{for a precise account}
for a precise definition of $\llbracket \, \, \rrbracket$ and an appropriate definition of
satisfaction, $\vDash$.

The constraint which turns a syntactic applicative structure, \textit{P}, into a
syntactic $\lambda$-algebra is
\begin{align*}
    \bm{\lambda} \vdash M = N \Rightarrow P \models \llbracket M \rrbracket = \llbracket N
    \rrbracket
\end{align*}

Or that every two $\lambda$-terms that are equal under $\bm{\lambda}$ are equal
under their intepretation within the $\lambda$-algebra. There is a close
relationship between $\lambda$-algebras and $CCCs$. Every $\lambda$-algebra can
be transformed into a $CCC$ with a reflexive object via a process known as the
\textit{Karoubi Envelope}.\todo{karoubi} Furthermore, every $CCC$ with a
reflexive object can be turned into a $\lambda$-algebra such that taking a
$\lambda$-algebra to a $CCC$ and back to $\lambda$-algebra again produces an
isomorphic $\lambda$-algebra. This indicates that every $\lambda$-algebra can
be obtained by a $CCC$ with a reflexive object. The results of \todo{turn into
theorem} can be interpreted in both directions of this transformation.
Intepreting within the context of the transformation from \textit{Karoubi
envelope} to CCC yields no sensical results and will be detailed in
\todo{appendix}. Interpreting in the other direction, however, is more useful.


A (locally small) $CCC$ with a reflexive object, $D$, with arrows $F : D
\rightarrow D^D$ and $G : D^D \rightarrow D$ can be turned into a
$\lambda$-algebra as follows. The underlying set of the $\lambda$-algebra are
the points to $D$, written $|D|$. The binary operation, $\star$, of the
generated $\lambda$-algebra that operates on points, $a$, $b$ to $D$ is as
follows:
\begin{align*}
    a \star b = eval \circ \langle \, F \times id \, \rangle \circ \langle a \,
    , \, b \,
    \rangle
\end{align*}

\todo{here} defines a semantic intepretation function for $\lambda$-terms for
which the triple of $( \, |D| \, , \, \star \, , \, \llbracket \rrbracket \,)$
is shown to be  a $\lambda$-algebra in.


With \todo{corollary} and the fact that the above transformation yields all
$\lambda$-models
fixed point theorem can be used to prove the first recursion theorem in every
$\lambda$-model. The proof proceeds as follows. In some cartesian closed
category with a point surjective arrow, \AgdaFunction{PS.arr}, the operation
constituting $\star$ in the generated $\lambda$-algebra can be written as
follows

\ExecuteMetaData[../agda/latex/ThesisExtra.tex]{extra-applicative-op}

Proving the first fixed point theorem in every $\lambda$-algebra amounts to
showing that the following type is inhabited
\ExecuteMetaData[../agda/latex/ThesisExtra.tex]{extra-ffpt}

Or that every point to the reflexive object has a fixed point under $\star$. To
prove this, the fixed point must be provided and a proof that it is a fixed
point. The fixed point can be constructed as follows beginning by precomposing the
point-surjective arrow with the given point to \verb|f|

\begin{AgdaMultiCode}
\ExecuteMetaData[../agda/latex/ThesisExtra.tex]{extra-x-def}
\\
This gives a point to $\texttt{D}^\texttt{D}$. This can be turned into a
endomorphism on \verb|D| as follows by passing it through two familiar
isomorphisms
\\
\ExecuteMetaData[../agda/latex/ThesisExtra.tex]{extra-x-isos}
\\
Lawvere's fixed point theorem can now be used to find a fixed point for
\verb|y|
\\
\ExecuteMetaData[../agda/latex/ThesisExtra.tex]{extra-x-law}
\\
The fixed point and proof can be extracted as follows
\\
\ExecuteMetaData[../agda/latex/ThesisExtra.tex]{extra-x-fix}
\\
\verb|fixedPoint| is the fixed point for $\star$. Now, it needs to be shown that
\verb|fixedPoint| is in fact a fixed point for \verb|f| under $\star$ i.e.
\\
\ExecuteMetaData[../agda/latex/ThesisExtra.tex]{extra-fix-proof}
The definition of $\star$ can be expanded to
\\
\ExecuteMetaData[../agda/latex/ThesisExtra.tex]{extra-expand-pre}
\ExecuteMetaData[../agda/latex/ThesisExtra.tex]{extra-expand}
\\
From this point the proof proceeds in a similar fashion to the proof of
lawvere's fixed point theorem. The key observation here is that any point to
\verb|X| can composed with \AgdaFunction{Ps.arr} to give a point to
$\texttt{X}^{\texttt{X}}$ which can be pushed through familiar isomorphisms to
give a endomorphism on \verb|X|. The application of the generated applicative
structure amounts to converting the left hand point of the operation into an
endomorphism and then composing with the right hand point.
\\
\ExecuteMetaData[../agda/latex/ThesisExtra.tex]{extra-bulk-proof}
\ExecuteMetaData[../agda/latex/ThesisExtra.tex]{extra-bulk-proof1}
\\
The right hand side of the outermost composition now matches the constructed
\verb|y| from earlier \\
\ExecuteMetaData[../agda/latex/ThesisExtra.tex]{extra-bulk-proof1}
\ExecuteMetaData[../agda/latex/ThesisExtra.tex]{extra-almost}
\ExecuteMetaData[../agda/latex/ThesisExtra.tex]{extra-almost1} \\
\verb|fixedPointProof| can now be applied to make use of lawvere's fixed point
theorem. \\
\ExecuteMetaData[../agda/latex/ThesisExtra.tex]{extra-almost1}
\ExecuteMetaData[../agda/latex/ThesisExtra.tex]{extra-end}
\end{AgdaMultiCode}

The above theorem shows that the syntactic applicative structure generated from
any CCC with a point surjective morphism has a fixed point theorem. The first
fixed point theorem can be recovered from the $\lambda$-calculus by considering
the syntactic term models. \todo{term models} have a two way implication.

The above proof raises further questions concerning categorical intepretations
of the untyped $\lambda$-calculus. $\lambda$-algebras are all the models that,
for their interpretation of $\lambda$-terms, satisfy the equations of
$\bm{\lambda}$. Other structures exist that aim to characterise all models for
which other equational theories hold of their intepretation of $\lambda$-terms.
For instance, $\lambda$-models are an extension of $\lambda$-algebras which also
satisfy the Meyer-Scott axiom of weak-extensionality. The class of CCCs that
give rise to the $\lambda$-models are those with a reflexive object but also
\textit{have enough points}. \todo{have enough points} Furthermore there are the
extensional $\lambda$-algebras that correspond the equational theory with the
satisfies the $\eta$ rule i.e.
\begin{align*} P \vDash \forall x (\lambda x . M
    x) = M
\end{align*}

The class of CCCs that give rise to these structures are those that \textit{have
enough points} but, instead of having a reflexive object $D$, instead have the
isomorphism $D \equiv D^{D}$.

Given that point surjectivity gives a fixed point theorem for any applicative
structure generated for it a natural question that arises is to what underlying
structure it might correspond. The notion is certainly weaker than that of
certainly weaker than that of a $\lambda$-algebra as the arrow from $D^{D}
\rightarrow D$ for a reflexive object is made use of when defining the semantic
intepretation function giving rise to $\lambda$-algebras. This thesis does not
provide a concrete answer to this question but provides some combinators that
are derivable in any applicative structure derived from a CCC with a point
surjective object. The derivations of the combinators will be presented
informally due to their similarity to the above derivations with formal proofs
being contained in the appendix.

These combinators utilise the fact that $a \star b =
\overline{uncurry \, (\varphi \circ a)} \circ b$. If  $a$ can be picked
then, any endomorphism on $D$ can be recovered. More precisely, for any $f:
D \rightarrow D$, this can be turned into a point to $D^D$ by pushing through
the other way in the $1 \times A \cong A$ isomorphism and exploiting the other
direction of the adjunction, here given the name curry i.e. $f' = curry \, (
\,\underline{f} \,) : 1 \rightarrow D^D$. The point-surjectivity of $\varphi$
can now be used to find the equivalent $u$ such that $\varphi \circ u = f'$.
Considering $u \star b$ for any $b$

\begin{align*}
    u \star b &= \overline{uncurry \, (\varphi \circ u \, )} \circ b \\
    &= \overline{uncurry \, ( \, f' \, ) } \circ b \\
    &= \overline{uncurry \, ( \, curry \, ( \, \underline{f} \, ) \, )} \circ b \\
    &= f \circ b
\end{align*}


A fairly easy combinator to construct is the identity combinator $\textbf{I} \, x =
x$ by taking $f$ in the above construction to be $id$ and calculating the
equivalent $u$.

Another useful combinator is the mockingbird or self-application operator,
$\textbf{M} x = x x$ by taking $f$ to be the following morphism of type $D
\rightarrow D$, $\textbf{M'} = eval \circ \langle \varphi \times id \rangle \circ
\delta$ and finding the equivalent $u$ and setting it to be \textbf{M}.

The final combinator a derivation is given for was found whilst attempting to
recover the \textbf{K} combinator and requires slightly more machinery.

Let $x = id : D \rightarrow D$, $y = curry \, ( \, \underline{x} \, ) : 1
\rightarrow D^D$ pick $z : 1 \rightarrow D$ s.t. $\varphi \circ z = y$ from the
point-surjectivity of $\varphi$ let $q = z \, \circ \, !_{D} : D \rightarrow D$
where $!_{D}$ is the terminal arrow from $D$. Taking $f$ as $q$ and deriving the
appropriate $u$ an interesting combinator is derived. Calling the appropriate
$u$, \textbf{F}

\begin{align*}
    \textbf{F} \star a \star b &= (\textbf{F} \circ a) \star b \\
    &= eval \circ \langle \varphi  \times id \rangle \circ \langle q \circ a , b
    \rangle \\
    &= eval \circ \langle \varphi \times id \rangle \circ \langle z \, \circ \, !_{D}
    \, \circ \, a , \, b \rangle \\
    &= eval \circ \langle \varphi \circ z \times id \rangle \circ \langle !_{D} \,
    \circ \, a \, , \, b \rangle \\
    &= \overline{uncurry \, ( \varphi \circ z \, )} \circ b \\
    &= id \circ b \\
    &= b
\end{align*}
i.e. \textbf{F} selects the second of its two arguments. With some work perhaps
more combinators could be derived and, potentially, they will form a complete
basis for combinatory logic. Even if not , it is interesting to understand to
which computational world point-surjectivity corresponds.


% -----------------------------------------------------------------------------

\chapter{Critical Evaluation}
\label{chap:evaluation}

% !TEX root = ./dissertation.tex

\section{Alternative Theorem Provers}
A seemingly abitrary decision was the choice of proof-assitant to use. Of the
plethora that exist, the three viable options for use in this project were Coq,
Agda and Isabelle. The reasons for choosing Agda were largely convenience. Agda
closely resembles in syntax and semantics strongly typed functional programming
languages like Haskell and therefore is more likely to resemble a more familiar
programming experience to computer scientists. The propositions-as-types
approach is taken to proving where proof objects are easily manipulable. Coq, is
a dependently typed functional programming language supported by INRIA and
originally developed by Gerard Huet and Thierry Coquand based on an alternative
to MLTT called the Calculus of Inductive Constructions. Proving in Coq does not
consist of pattern matching and manipulating proof objects but instead through
the refinement of goals through \textit{tactics}. Tactics 
\section{Limitations}
As mentioned earlier in \todo{section}, a setoid based approach was taken towards
modelling categories where categories were parameterised by an equivalence
relation on morphisms. This limitation arose due to the inability to represent
quotient sets within the type theory of plain Agda. This was a limitation in
several different ways. As previously mentioned, for arbitrary equivalence
relations, there is no support for indercinability of identicals and congruence
must therefore be proved for every type individually. This did not prove to be a
particularly annoying component of proving within the scope of this thesis.
This, however, is primarily due to not working in any instances of categories
that required working with an equivalence relation of morphisms. It would be
difficult to take the version of Lawvere's theorem defined within this thesis
and apply it to a non-trivial category. One example of this would be to, within
Agda, construct the category $\bm{CPO}_{\bot}$\todo{explain}, which is a CCC with a reflexive
object, to examine the implications of the results in \ldots\todo{here}


\subsection{Constructive Category of Sets}
Another, more restrictive limitation is the lack of a category of sets to work
within. In \ldots \todo{section on cantor's theorem}, rather than working in a
real category of sets, an analog to the theorem was proven in a category of
small types. This was not a faithful interpretation as, within type theory, the
notion of typing is \textit{external} to the theory, whereas a key notion within
set theory is that the membership relation is internal to the theory. This is
what allows the notion of powerset to be formalised and thus takes the
formulation presented in \ldots to Cantor's theorem. To construct a category of
sets it is necessary to work within a different type theory.

There are various different type theories that could be worked within.
observational blah blah blah.

A type theory which is supported by Agda and provides a solution to both of the
problems outlined above is the new and radically different type theory known as
Homotopy Type Theory.

\subsection{Homotopy Type Theory}

Homotopy type theory is an extension of MLTT in which the
higher dimensional structure of the equality type is embraced. Homotopy type
theory rejects \textsf{Axiom K} which implies UIP. By doing so the so-called higher
homotopy structure of the identity type can be exploited. This is achieved by
viewing equality between two types as a path in space between two points. Two
different paths can be considered as well as homotopies (continuous
deformations) between them. Through rejecting \textsf{Axiom K}, it is no longer
possible to prove that \textsf{refl} is the only inhabitant of the identity
type. An upside of this is that new inhabitants of the equality type for a given
type can now be introduced without inconsistency. It is possible to imagine
that, to create a quotient type, new equalities can be introduced corresponding
directly to the equivalence classes desired. Types with these non-trivial
equalities are known as higher inductive types.

Rejecting \textsf{Axiom K} by itself provides benefits to the type theory but
another significant advantage can be gained, namely the introduction of the
univalence axiom. The univalence axiom, inconsistent with
\textsf{Axiom K}, is an axiom which states that equivalence is equivalent to
equality.

\begin{align*}
    A = B \simeq A \simeq B
\end{align*}

jere, equivalence is meant in the category theoretic sense in that
two types $A$ and $B$ belonging to some universe $U$ are equivalent if there
exists an arrow $f : A \rightarrow B$ with both a left inverse and right
inverse. The univalence axiom captures an informal mathematical practice of
equating two isomorphic structures, allowing the informal reasoning in
mathematics to be done within the type theory. Univalence simplifies the task of
working with higher inductive types and allows mathematics such as Category
theory (very much dependent on classifying structures up to isomophism) and
homotopy theory to be embedded succinctly within type theory. By postulating
univalence it is possible to recover a classical principle of function
extensionality, something that is not valid within plain MLTT.

Within homotopy type theory it is possible, as mentioned prior, to quotient your
types by abitrary equivalence relations as a higher inductive type.
\textsf{Axiom J} without \textsf{Axiom K} ensures that these higher inductive
types behave the same as \textsf{refl}, meaning that indecernability of identicals
is preserved. This allows categories to be defined with propositional equality
without the additional requirement of congruence for the equivalence relation.
Within Homotopy type theory it is possible to define a set type which
corresponds to the rules of ZFC with a coherent notion of set membership. This
is outlined in \todo{here}. Using this construction, a coherent category of sets
could be produced within the type theory and Lawvere's fixed point theorem could
be applied appropriately.

Univalence is posited as an axiom within Homotopy Type theory but work has been
done recently to provide a constructive interpretation of the univalence axiom
\todo{cubical}.  Support for Cubical type theory exists in Agda as of version
2.6 providing support for higher inductive types and an equality path type.

\section{Importance of Theorem Proving}
Part of the  work done during this thesis was to investigate how computer-aided
mathematics is done and whether it is worth being pursued in mainstream
mathematics. With respect to the latter question, throughout this thesis it has
become clear that, although there are certainly limitations and problems,
computer-aided theorem proving is a positive step forward for mathematics. Much
mathematical reasoning is done informally and even in published proofs many
steps are left for the reader to internally fill in. There is no shortage of
mathematical proofs that have been published only for mistakes to have been
found. The Russian Mathematician Vladimir Voevodsky, a central figure behind the
development of Homotopy type theory, began his career in algebraic topology. A
main motivation behind his development of Homotopy type theory was due to
mistakes in published papers throughout the 1970s within algebraic topology.
Homotopy type theory was aimed at allowing mathematicians to use informal
reasoning within a vigorous setting. More can be found of voevodsky's opinions
within \todo{voevodsky}

In attempts to investigate the relationship between the $\lambda$-calculus and
Lawvere's fixed point theorem, two informal attempts to outline the relation
were found. These attempted proofs both had inaccuracies which will be outlined
shortly. Through working in a highly rigorous setting, there is very little
doubt that relation outlined within this thesis is incorrect. Although not all
the relationships have been investigated end to end, the tricky area of the proof
(i.e. the algebraic manipulation) has been formalised and checked. The work
formalising Lawvere's theorem was vital to observing the connection between the
$\lambda$-calculus and the theorem. In theorem proving, none of the details can
be elided so a subtle and deep understanding of the intricacies of the theorem
is developed. It is by paying close attention to the types of terms that the
below errors were observed.

\subsubsection{The CCC generated from a simply-typed $\lambda$-calculus}

A recent Ph.D. thesis, \textit{Category Theory and the Lambda Calculus} by Mario
Rom\'an Garcia provides an attempt to integrate Lawvere's theorem into the
theory of the untyped $\lambda$-calculus as follows

Garcia uses Joachim Lambek's observation that every simply-typed
$\lambda$-calculus has an associated cartesian closed category which is obtained
by considering the category in which the tp 





$D \equiv D^D$

% $\textbf{CPO}_{\bot}$ the category of cpos with $\bottom$ objects are cpos with
% least element, arrows are scott continuous function, define a ccc and exists an
% object with a retraction therefore corollary therefore every endomap has a fixed
% point. Every CPO with retraction gives rise to $\lambda$-model underlying set is
% set of CPO and operation is F(u)(v) where F : D -> (D -> D) and u , v are elems
% of set 


CCC from simply typed $\lambda$-calculus
\section{future work}
\subsection{Linenbaum-Tarski Algebras}
godels incompleteness theorem
\subsection{Lambda calculus and topoi}
tarki's theorem



% -----------------------------------------------------------------------------

\chapter{Conclusion}
\label{chap:conclusion}

% !TEX root = ./dissertation.tex

{\bf A compulsory chapter,     of roughly $5$ pages}
\vspace{1cm}
\noindent

\ExecuteMetaData[../agda/latex/Thesis.tex]{plus}

The concluding chapter of a dissertation is often underutilised because it
is too often left too close to the deadline: it is important to allocation
enough attention.  Ideally, the chapter will consist of three parts:


\begin{enumerate}
\item (Re)summarise the main contributions and achievements, in essence
      summing up the content.
\item Clearly state the current project status (e.g., ``X is working, Y
      is not'') and evaluate what has been achieved with respect to the
      initial aims and objectives (e.g., ``I completed aim X outlined
      previously, the evidence for this is within Chapter Y'').  There
      is no problem including aims which were not completed, but it is
      important to evaluate and/or justify why this is the case.
\item Outline any open problems or future plans.  Rather than treat this
      only as an exercise in what you {\em could} have done given more
      time, try to focus on any unexplored options or interesting outcomes
      (e.g., ``my experiment for X gave counter-intuitive results, this
      could be because Y and would form an interesting area for further
      study'' or ``users found feature Z of my software difficult to use,
      which is obvious in hindsight but not during at design stage; to
      resolve this, I could clearly apply the technique of Smith [7]'').
\end{enumerate}


% =============================================================================

% Finally, after the main matter, the back matter is specified.  This is
% typically populated with just the bibliography.  LaTeX deals with these
% in one of two ways, namely
%
% - inline, which roughly means the author specifies entries using the
%   \bibitem macro and typesets them manually, or
% - using BiBTeX, which means entries are contained in a separate file
%   (which is essentially a database) then imported; this is the
%   approach used below, with the databased being dissertation.bib.
%
% Either way, the each entry has a key (or identifier) which can be used
% in the main matter to cite it, e.g., \cite{X}, \cite[Chapter 2}{Y}.

\backmatter

\bibliography{dissertation}

% -----------------------------------------------------------------------------

% The dissertation concludes with a set of (optional) appendicies; these are
% the same as chapters in a sense, but once signaled as being appendicies via
% the associated macro, LaTeX manages them appropriatly.

\appendix

\chapter{The Karoubi Map and Lawvere's Fixed Point Theorem}
\label{appx:karoubi}

Content which is not central to, but may enhance the dissertation can be
included in one or more appendices; examples include, but are not limited
to

\begin{itemize}
\item lengthy mathematical proofs, numerical or graphical results which
      are summarised in the main body,
\item sample or example calculations,
      and
\item results of user studies or questionnaires.
\end{itemize}

\noindent
Note that in line with most research conferences, the marking panel is not
obliged to read such appendices.

% =============================================================================

\end{document}
