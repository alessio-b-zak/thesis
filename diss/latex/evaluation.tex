% !TEX root = ./dissertation.tex
\section{Alternative Theorem Provers}
A seemingly abitrary decision was the choice of proof-assitant to use. Of the
plethora that exist, the three viable options for use in this project were Coq,
Agda and Isabelle. The reasons for choosing Agda were largely convenience. Agda
closely resembles in syntax and semantics strongly typed functional programming
languages like Haskell and therefore is more likely to resemble a more familiar
programming experience to computer scientists. The propositions-as-types
approach is taken to proving where proof objects are easily manipulable. Coq, is
a dependently typed functional programming language supported by INRIA and
originally developed by Gerard Huet and Thierry Coquand based on an alternative
to MLTT called the Calculus of Inductive Constructions. Proving in Coq does not
consist of pattern matching and manipulating proof objects but instead through
the refinement of goals through \textit{tactics}. Tactics 
\section{Homotopy Type Theory}
As mentioned earlier in the thesis, a setoid based approach was taken towards
modelling categories where categories were parameterised by an equivalence
relation on morphisms. This limitation arose due to the inability to represent
quotient sets within the type theory of plain Agda. A very modern advancement in
type theory enables quotient types, the analog of quotient sets, to be
constructed directly.

Quotient types

Homotopy type theory is an extension of MLTT in which the
higher dimensional structure of the equality type is embraced. Homotopy type
theory rejects Axiom K which implies UIP. By doing so the so-called higher
homotopy structure of the identity type can be exploited.

without k
higher inductive types
univalence
quotient types
cubical type theory agda
cubical agda

\section{Future Work}
topos theory

\subsection{Constructive Category of Sets}
Intepreting Lawvere's fixed point theorem within the category of sets to
produce the Russell style paradoxes and Cantor's theorem proves tricky within
Agda. Within normal MLTT a constructive category of sets is

sets in mltt

ects

zfc in hott

\subsection{Linenbaum-Tarski Algebras}
godels incompleteness theorem
tarki's theorem
\subsection{Lambda calculus and topoi}
\section{Importance of Theorem Proving}
problems with other proofs
