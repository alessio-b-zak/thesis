% !TEX root = ./dissertation.tex

\section{Alternative Theorem Provers}
\label{sec:altthmprov}
A relatively arbitrary decision was the choice of proof-assitant to use. Of the
plethora that exist, the two most popular in use today are Coq
\cite{coq}
and Agda. The reasons for choosing Agda were largely convenience. Agda
closely resembles strongly typed functional programming languages like Haskell
in syntax and therefore is more likely to resemble a more familiar programming
experience to computer scientists. The propositions-as-types approach is taken
to proving where proof objects are easily manipulable. Coq is a dependently
typed functional programming language supported by INRIA and originally
developed by Gerard Huet and Thierry Coquand based on an alternative to MLTT
called the Calculus of Inductive Constructions \cite{pfenning1989inductively}. Proving in Coq does not consist
of pattern matching and manipulating proof objects but instead through the
refinement of goals through \textit{tactics}. Tactics are procedures which take
a goal (proof) and apply a deductive step in reverse to produce subgoals which
themselves must be solved. This procedure continues until all goals have been
satisfied and the proof is complete. In this way, the direct manipulation of
terms can be avoided and large steps of proofs can be automated. Users are able
to define their own tactics allowing for faster and more automated proofs. This
style is not without its disadvantages. Coq proofs consist of a list of
the tactics used. These can be difficult to read and can require more cognitive
effort to understand. The timeframe for this project favoured the more intuitive
and familiar interface Agda offered. Declarative style proofs i.e.
\AgdaFunction{begin}-style, however, are not well supported in Agda, with no editor
integration.

\section{Limitations}
As mentioned earlier in Section \ref{section:setoids}, a setoid based approach
was taken towards modelling categories where categories were parameterised by an
equivalence relation on morphisms. This limitation arose due to the inability to
represent quotient sets within the type theory of plain Agda. This was a
limitation in several different ways. As previously mentioned, for arbitrary
equivalence relations, there is no support for indiscernibility of identicals and
congruence must therefore be proved for every type individually. This did not
prove to be a particularly annoying component of proving within the scope of
this thesis.  This, however, is primarily due to not working in any instances of
categories that required working with an equivalence relation of morphisms. It
would be difficult to take the version of Lawvere's theorem defined within this
thesis and apply it to a non-trivial category.

\subsection{Constructive Category of Sets}
\label{section:sets}
Another, more restrictive limitation is the lack of a category of sets to work
within. In Section \ref{section:cantor}, rather than working in a
real category of sets, an analog to the theorem was proven in a category of
small types. This was not a faithful interpretation as, within type theory, the
notion of typing is \textit{external} to the theory, whereas a key notion within
set theory is that the membership relation is internal to the theory. This is
what allows the notion of powerset to be formalised and thus takes the
formulation presented in Section \ref{section:cantor} to a faithful
interpretation of Cantor's theorem. To construct a category of
sets it is necessary to work within a different type theory.

A type theory which is supported by Agda and provides a solution to both of the
problems outlined above is the new and radically different type theory known as
homotopy type theory.

\subsection{Homotopy Type Theory}
\label{section:hott}
Homotopy type theory (HoTT) \cite{hottbook} is an extension of MLTT in which the
higher dimensional structure of the equality type is embraced. homotopy type
theory rejects \textsf{Axiom K} which implies uniqueness of identity proofs. By doing so the so-called higher
homotopy structure of the identity type can be exploited. This is achieved by
viewing equality between two types as a path in space between two points. Two
different paths can be considered as well as homotopies (continuous
deformations) between them. Through rejecting \textsf{Axiom K}, it is no longer
possible to prove that \textsf{refl} is the only inhabitant of the identity
type. An upside of this is that new inhabitants of the equality type for a given
type can now be introduced without inconsistency. It is possible to imagine
that, to create a quotient type, new equalities can be introduced corresponding
directly to the equivalence classes desired. Types with these non-trivial
equalities are known as higher inductive types.

Rejecting \textsf{Axiom K} by itself provides benefits to the type theory but
another significant advantage can be gained, namely the introduction of the
univalence axiom. The univalence axiom, inconsistent with
\textsf{Axiom K}, is an axiom which states that equivalence is equivalent to
equality.

\begin{align*}
    (A = B) \simeq (A \simeq B)
\end{align*}

Here, equivalence is meant in the category theoretic sense in that
two types $A$ and $B$ belonging to some universe $U$ are equivalent if there
exists an arrow $f : A \rightarrow B$ with both a left inverse and right
inverse. The univalence axiom captures an informal mathematical practice of
equating two isomorphic structures, allowing the informal reasoning in
mathematics to be done within the type theory. Univalence simplifies the task of
working with higher inductive types and allows mathematics such as Category
theory (very much dependent on classifying structures up to isomophism) and
homotopy theory to be embedded succinctly within type theory. By postulating
univalence it is possible to recover a classical principle of function
extensionality without postulation.

Within homotopy type theory it is possible, as mentioned prior, to quotient your
types by abitrary equivalence relations as a higher inductive type.
\textsf{Axiom J} without \textsf{Axiom K} ensures that these higher inductive
types behave the same as \textsf{refl}, meaning that indecernability of identicals
is preserved. This allows categories to be defined with propositional equality
without the additional requirement of congruence for the equivalence relation.
Within homotopy type theory it is possible to define a set type which
corresponds to the rules of ZFC with a coherent notion of set membership. This
is outlined in \textbf{Chapter 10} of the HoTT book \cite{hottbook} . Using this construction, a coherent category of sets
could be produced within the type theory and Lawvere's fixed point theorem could
be applied appropriately.

Univalence is posited as an axiom within homotopy Type theory but work has been
done recently to provide a constructive interpretation of the univalence axiom
in a type theory known as Cubical type theory \cite{cohen2016cubical}.  Support for Cubical type theory exists in Agda as of version
2.6 providing support for higher inductive types and an equality path type.

From the above remarks it is clear that almost all the limitations of the
current formalisation of categories would be solved by moving towards a homotopy
type theory setting, hopefully resulting in a simpler and more satisfying
proving experience.
\section{Importance of Theorem Proving}
Part of the  work done during this thesis was to investigate how computer-aided
mathematics is done and whether it is worth being pursued in mainstream
mathematics. With respect to the latter question, throughout this thesis it has
become clear that, although there are certainly limitations and problems,
computer-aided theorem proving is a positive step forward for mathematics. Much
mathematical reasoning is done informally and even in published proofs many
steps are left for the reader to internally fill in. There is no shortage of
mathematical proofs that have been published only for mistakes to have been
found. The Russian Mathematician Vladimir Voevodsky, a central figure behind the
development of homotopy type theory, began his career in algebraic topology.  In
an open letter entitled \textit{The Origins and Motivations of Univalent
Foundations} as part of the  Institute for Advanced Study Letter published in 2014 \cite{sorlin2014iasthe}, Voevodsky outlined his
motivations for creating more advanced tools in which to \textit{do} mathematics.
He outlines the mistakes and fear at discovering mistakes in theorems he had
published and minor conflicts between mathematicians over the existence of
counter examples. The motivations behind the programme of theorem proving can be
neatly summed up by Voevodsky:

\begin{displayquote}
    \textit{And I now do my mathematics with a proof assistant. I have a lot of wishes
    in terms of getting this proof assistant to work better, but at least I
    don’t have to go home and worry about having made a mistake in my work. I
    know that if I did something, I did it, and I don’t have to come back to it
    nor do I have to worry about my arguments being too complicated or about how
    to convince others that my arguments are correct. I can just trust the
    computer.}
\end{displayquote}

In attempts to investigate the relationship between the $\lambda$-calculus and
Lawvere's fixed point theorem, two informal attempts to outline the relation
were found. These attempted proofs did not fully link the untyped
$\lambda$-calculus to Lawvere's theorem. Through working in a highly rigorous
setting, there is no reason to doubt that the relation ship outlined within this thesis is
correct. Although not all the relationships have been investigated end-to-end,
the tricky aspect of most proofs (i.e. the algebraic manipulation) has been
formalised and checked. The work formalising Lawvere's theorem was vital to
observing its connection to the $\lambda$-calculus. In
theorem proving, none of the details can be elided so a subtle and deep
understanding of the intricacies of the theorem is developed. Below are the two
proofs that were found, with commentary on their aims and developments.

\subsection{Prior Unifications of the Untyped $\lambda$-calculus and
Lawvere's Theorem}
\label{quote:prior}
\subsubsection{The CCC generated from a simply-typed $\lambda$-calculus}


A recent Bachelors thesis, \textit{Category Theory and the Lambda Calculus} by Mario
Rom\'an Garcia \cite{garcia2018thesis} provides an attempt to integrate Lawvere's theorem into the
theory of the untyped $\lambda$-calculus.

Garcia uses Joachim Lambek's observation that every simply-typed
$\lambda$-calculus has an associated cartesian closed category which is obtained
by considering the objects of the category to be the types of the
$\lambda$-calculus and the morphisms to be the functions between types. Lambek
showed in \cite{lambek1985cartesian} that these categories were cartesian closed. Through the
familiar argument that the untyped $\lambda$-calculus can
be viewed as a simply typed calculus i.e.  that untyped is uni-typed - the
untyped $\lambda$-calculus can be viewed as a simply typed $\lambda$-calculus
with one type, \textsf{D}. Garcia uses this argument and also provides a
retraction pair $r : \textsf{D} \rightarrow (\textsf{D} \rightarrow
\textsf{D})$, $s : (\textsf{D} \rightarrow \textsf{D}) \rightarrow \textsf{D}$.
Following this line of reasoning, Garcia establishes that considering the
untyped-as-unityped $\lambda$-calculus as a cartesian closed category yields a
category with a reflexive object, \textsf{D} where terms can be considered as
morphisms via the retraction map. By Lawvere's fixed point theorem every
$\lambda$-term, $d : \textsf{D}$ , has a point $p: 1 \rightarrow \textsf{D}$
such that $d \circ p = p$. Garcia claims in \textbf{Corollary 4.20} that this
proves that every term in the untyped $\lambda$-calculus has a fixed point. This
argument fails on a crucial point. The cartesian closed category generated from
the untyped-as-unityped $\lambda$-calculus has a single object. By the
definition of a cartesian closed category this object \textit{must} be the
terminal object. To recap, the definition of a terminal object is that for all
objects in the category there is a unique arrow from the object to the terminal
object. This condition applies to the terminal object itself and therefore there
is a unique arrow from the terminal object to the terminal object. By the
definition of a category every object must have a terminal object and therefore
the only arrow from the terminal object to itself can be the identity arrow.
With this reading it is clear that there can be no syntactic interpretation of
this arrow that yields the terms of the untyped $\lambda$-calculus. A key
component of result in Section \ref{quote:lambda}  was that it gave a fixed point
theorem in every $\lambda$-algebra which allowed a route back to the untyped
$\lambda$-calculus.

\subsubsection{Interpretation in the reflexive cpo in the CCC -
\textbf{CPO}$\bot$}
The argument that follows is not incorrect but is derivable as a corollary via
the result in this thesis.

The argument from a note published by Dan Frumin and Guillaume
Massas \cite{frumin2017note}. Their arguments draw on material from throughout the Barendregt
\cite{barendregt1992lambda} which will be made clear when necessary. They begin
their argument by considering the category \textbf{CPO}$_{\bot}$. The objects of
this category are directed complete partial orders (cpos) with a least element
and arrows are scott-continuous functions. \textbf{Theorem 1.2.16} in the
Barendregt shows that this category is cartesian closed. Frumin and Massas begin
by stating without proof that there is a reflexive object in
\textbf{CPO}$_{\bot}$.
Using Lawvere's theorem, Frumin and Massas conclude that every endomap on this
object has a fixed point in the categorical sense. \textbf{Section 5.4} of the
Barendregt outlines how every reflexive cpo with maps $F : D \rightarrow (D
\rightarrow D)$ and $G : (D \rightarrow D) \rightarrow D$ induces a
$\lambda$-model. The underlying set of the $\lambda$-model is the underlying set
of the cpo and the binary operation $\bullet : D \times D \rightarrow D$ is
defined on $x , y \in D$ as \begin{align*} x \bullet y = F(x)(y) \end{align*}
Frumin and Massas argue to the effect that, given that every endomap has a
categorical fixed point, that scott-continuous function $f : D \rightarrow D$
must have a fixed point. This is valid reasoning as the points to a cpo
correspond to the elements of the cpo and therefore every endomorphism on $D$
having a fixed point $p : 1 \rightarrow D$ corresponds to every scott-continuous
function on $D$ having a fixed point. Whilst the reasoning of this argument is
correct what has been proven is that a particular model of the
$\lambda$-calculus has fixed points. Whilst suggestive there is no guarantee
that it will be possible to translate this back to the syntax of the untyped
$\lambda$-calculus. The extension to work with abstract cartesian closed
categories with a reflexive object is a key component of relating the result
back the syntax of the untyped $\lambda$-calculus. This could instead be
considered an application of Lawvere's theorem to domain theory.

\section{Future Work}
There is much scope to extend the work done within this thesis. There are a
significant number of applications of Lawvere's theorem which have yet to be
explored within a computational setting. The formalised proof of Lawvere's
theorem could be used to instantiate these paradoxes if the appropriate
categories are constructed within the type theory. Many of the applications and
paradoxes are within the category of sets or a set-like category.


With this in mind the most practical next step would be to reformalise the work
done within this thesis within a language that supports homotopy type theory
taking advantage of univalence and higher inductive types. There is much
groundwork in place to assist in this endeavour, \textbf{Chapter 9} of
the HoTT book details how category theory can be constructed within HoTT and
there exist implementations of this within Cubical Agda \cite{iverson2018cat}. The
process of reimplementing the proofs within this thesis in a cubical setting
would be an interesting process itself to examine. Working within Agda during
this thesis was at times tedious but ultimately an intellectually tractable
process. HoTT represents a significant shift in the viewpoint of theorem proving
and its practicality is still worth being assessed. Does the cognitive overhead
outweigh the practical advantages that univalence provides. Once reformalised, a
it would be possible to work within a category of sets where the aforementioned
questions could be formalised.


As was mentioned at the end of Section \ref{quote:lambda} to which computational
calculi point surjectivity corresponds is left as an open question. The
structure in question yields a fixed point theorem and several interesting
combinators. Whether or not point-surjectivity yields a functionally complete
set of combinators in its generated applicative structure is predicated on the
categorical analog of models of combinatory logic, combinatory algebras.

\subsection{Combinatory Algebras and Lawvere's Theorem}
\label{section:combin}
Longo and Moggi \cite{longo1990category} provide the categorical analogue of the combinatory models, the
class of models of combinatory logic. A combinatory algebra is a 4-tuple
$(A, \bullet, k, s)$ where $A$ is a set, $\bullet$ is a binary operation on
$A$, and $k$ and $s$ are distinguished elements of $A$ that satisfy

\begin{align*}
    &k \bullet x \bullet y = x \\
    & s \bullet x \bullet y \bullet z = x \bullet z \bullet (y \bullet z)
\end{align*}

An interesting observation is that, to model combinatory algebras, the category
does not have to be cartesian closed.  This is noteworthy as Lawvere's theorem
also has an interpretation solely within a cartesian category. Longo and Moggi
establish that all combinatory algebras can be obtained from a cartesian
category with an object $U$ such that there exist arrows $f : U \times U
\rightarrow U$ and $g : U \rightarrow U \times U$ such that $g \circ f = id$
\textit{and} that there exists a K-universal arrow $u :U \times U
\rightarrow U$.

In a cartesian category, an arrow $u : X \times Y \rightarrow Z$ is K-universal
if for all $f : X \times Y \rightarrow Z$ there exists a unique $s : X
\rightarrow X$ such that $f = u \circ \langle \, s \times id \, \rangle$.

By transporting the definition of Lawvere's theorem outlined within this thesis
through the adjunction between the functors $-\times Y$ and $-^{Y}$ a statement
is obtained within cartesian categories. The equivalent theorem proceeds as
follows and is outlined in \textbf{Theorem 2.2.} of \cite{lawvere1969diagonal}.
In a cartesian category if there exists some $g : A \times A \rightarrow Y$ such
that, for all $f : A \rightarrow Y$ there exists an $x : 1 \rightarrow A$ such
that for all $a : 1 \rightarrow A$

\begin{align*}
    f \circ a = g \circ \langle a , x \rangle
\end{align*}

The interplay between the two definitions in this section will do much to
enlighten the relationship between functional completeness and Lawvere's fixed
point theorem and is a topic for future study.
